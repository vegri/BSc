Im Rahmen dieser Arbeit wurden zwei unterschiedliche Themen bearbeitet. Einerseits wurde das
Wachstum von Gold auf einem (0001)-orientiertem Rheniumkristall untersucht, da Gold als
inertes Material wiederum wünschenswerte Eigenschaften als Substrat für organische Moleküle
aufweist. Andererseits wurde eine Apparatur zur Spraydeposition aufgebaut und getestet, mit
dessen Hilfe die genannten Moleküle ohne Verdampfen, sondern nur mittels Druckgefälle innerhalb
des Aufbaus auf das Substrat aufgebracht werden können.
\\
Zunächst wurde die Oberfläche des Rheniumkristalls mittels LEED und Rastertunnelmikroskopie
untersucht. Zwar gab es in der Mitte der verwendeten Probe Verunreinigungen oder andere Defekte,
dessen unperiodische Struktur durch LEED festgestellt werden konnte, doch waren die äußeren Bereiche
des Kristalls trotz leichter Streckung der Gitterstruktur, mit der die Oberfläche eher einem bcc-
statt einer hcp-Gitter glich, als Substrat prinzipiell geeignet. In den STM-Bildern konnten ebene
Terrassen mit etwa \SI{35}{nm} Breite beobachtet werden, auf denen ein Wachstum von Gold gut
beobachtet werden sollte.
\\
In verschiedenen Schichtdicken aufgdampftes Gold wurde ebenfalls mit LEED und STM untersucht. Es
wurde ein Stransky-Krastanov-Wachstum, d.h. ein Inselwachstum beobachtet. In den LEED-Aufnahmen
ergab dies bis zu den Schichtdicken von etwa zehn Monolagen diffuse Beugungsreflexe, was auf eine
wenig ausgeprägte Periodizität der Oberfläche schließen lässt. Ab zehn Monolagen wurden die
LEED-Reflexe ausgeprägter und das Untergrundleuchten schwächer, es ergab sich eine hexagonale
Periodizität der Goldoberfläche. Als mögliches Substrat für organische Moleküle, bei der große,
flache Inseln des Substrats zur Ermittlung der Anordnung und Struktur der Moleküle von Vorteil sind,
wurden von daher 20 und 30 Monolagen betrachtet. Da sich das Inselwachstum beider Schichtdicken was Ebenmäßgkeit und Größe
der Inseln - etwa \SI{35}{nm} Durchmesser der größten, 
\SI{7}{nm} Durchmesser der kleinsten, obersten Inseln - anging nicht signifikant voneinander
unterschied, wurden danach 20 Monolagen noch einmal bei einem Filamentstrom von
\SI{2,0}{A} bzw etwa \SI{650}{K} getempert und mit den ungetemperten 20 Monolagen verglichen. Es
ergab sich eine Glättung der Oberfläche, d.h. die Inseln wurden größer mit einem Inseldurchmesser
von etwa SI{15}{nm} bis SI{35}{nm}. Verschiedene Höhenprofile entlang der Inselb ergaben ebenso,
dass die Inseln sehr eben sind.
\\
Somit würden 20 Monolagen auf dem Rheniumsubstrat bei etwa \SI{650}{K} getempert ausreichen, um
genügend große Inseln zu erhalten, auf denen organische Moleküle aufbracht und untersucht werden
können. Eventuell könnte man durch Tempern bei höheren Temperaturen oder über einen längeren
Zeitraum eine weitere Glättung der Oberfläche erzielen.
\\




% Gold auf Re:
% 

% 
% 
% SB:
% -Glasröhrchen in ventil, um mehr flüssigkeit reintun zu können (wenn über langen zeitraum viel riengemacht
% werden muss), spritzschutz
% -ausrichtung substrat in molekülkammer - sichtfenster weg! markierung an magnet?
% -konus oben spitzer machen? damit keine tropfenbildung auf "`terasse"'? oder konus oben vergrößern?
% -filtern der lösung oder verdünnen? geht filtern auch mit größeren molekülen?
% -zum säubern muss ventil abgeschraubt werden, da nicht erreichbares reservoir direkt über öffnung
% -evtl weg verkürzen mit kürzerem zwischenstück
% -weitere test mit quarzwaage zur skalierung? bzw groben abschätzung, da jedes molekül anderes
% gewicht



Der Test der Spraydepositionsapparatur erbracht positive Ergebnisse, obgleich noch einige
Optimierungen des Aufbaus in Angriff genommen werden sollten.
%  Ziel war, organische Moleküle auf
% einem Substrat zu deponieren, ohne diese dabei wie bisher verdampfen, d.h. erhitzen zu müssen 
Zuletzt konnte mit der Quarzwaage anstelle des
Glasträgers gezeigt werden, dass genügend Moleküle auf der Waage ankommen, um die Frequenz des Schwingquarzes zu ändern. Daher sollten auch
bei einem Aufbau auf der Molekülkammer der UHV-Apparatur auf diese Weise aufgebrachte Moleküle auf
dem Substrat zum Liegen kommen. Interessant wäre festzustellen, wieviele Monolagen bei wieviel
Skalenteilen des Schwingquarzes auf dem Quarz bzw. auf dem Substrat haften bleiben. Dies hängt
natürlich von der Masse der verwendeten Moküle ab, sodass eine Kalibration des Quarzes nur für das
jeweilige aufgebrachte Molekül gilt. Andererseits kann eventuell über einen Massenvergleich zwischen
zur Kalibration benutztem Molekül und unbekanntem Molekül die Schichtdicke pro Skalenteile
abgeschätzt werden.\\
Nicht zu vergessen ist hier natürlich das ebenfalls aufgesprühte Lösungsmittel, das im
Allgemeinen mit der Zeit auch wieder verdampft. Da das Lösungsmittel in der Regel nicht
mituntersucht werden soll, ist ebenfall interessant herauszufinden, wie lange man nach dem
Aufbringen der Moleküle etwa abwarten muss, bis ein Großteil des Lösungsmittels vom Substrat
verdampft ist.\\

Leider konnte nicht, wie ursprünglich geplant, der Aufbau nach dem Test noch auf die Molekülkammer
der UHV-Apparatur aufgesetzt und die auf diese Weise auf den mit Gold präparierten Rheniumkristall
aufgebrachten Moleküle mit Hilfe der Rastertunnelmikroskopie untersucht werden.
