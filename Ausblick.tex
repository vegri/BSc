Gold auf Re:





SB:
-Glasröhrchen in ventil, um mehr flüssigkeit reintun zu können (wenn über langen zeitraum viel riengemacht
werden muss), spritzschutz
-ausrichtung substrat in molekülkammer - sichtfenster weg! markierung an magnet?
-konus oben spitzer machen? damit keine tropfenbildung auf "`terasse"'? oder konus oben vergrößern?
-filtern der lösung oder verdünnen? geht filtern auch mit größeren molekülen?
-zum säubern muss ventil abgeschraubt werden, da nicht erreichbares reservoir direkt über öffnung
-evtl weg verkürzen mit kürzerem zwischenstück
-weitere test mit quarzwaage zur skalierung? bzw groben abschätzung, da jedes molekül anderes
gewicht



Der Test des im Rahmen dieser Bachelorarbeit gebauten Sprühbeschichters verlief alles in allem
positiv. Zuletzt konnte mit der Quarzwaage anstelle des Glasträgers gezeigt werden, dass genügend
Moleküle auf der Waage ankommen, um die Frequenz des Schwingquarzes zu ändern. Daher sollten auch
bei einem Aufbau auf der Molekülkammer der UHV-Apparatur auf diese Weise aufgebrachte Moleküle auf
dem Substrat zum Liegen kommen. Interessant wäre festzustellen, wieviele 

Leider konnte nicht, wie ursprünglich geplant, der Aufbau nach dem Test noch auf die Molekülkammer
der UHV-Apparatur aufgesetzt und die auf diese Weise auf den mit Gold präparierten Rheniumkristall
aufgebrachten Moleküle mit Hilfe der Rastertunnelmikroskopie untersucht werden.
