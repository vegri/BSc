Im Rahmen dieser Arbeit wurden zwei unterschiedliche Themen bearbeitet. Einerseits wurde das
Wachstum von Gold auf einem (0001)-orientiertem Rheniumkristall untersucht, da Gold als
inertes Material wiederum wünschenswerte Eigenschaften als Substrat für organische Moleküle
aufweist. Andererseits wurde eine Apparatur zur Spraydeposition aufgebaut und getestet, mit
dessen Hilfe die genannten Moleküle ohne Verdampfen, sondern nur mittels Druckgefälle innerhalb
des Aufbaus auf das Substrat aufgebracht werden können.
\\
Zunächst wurde die Oberfläche des Rheniumkristalls mittels LEED und Rastertunnelmikroskopie
untersucht. Zwar gab es in der Mitte der verwendeten Probe Verunreinigungen oder andere Defekte,
dessen unperiodische Struktur durch LEED festgestellt werden konnte, doch waren die äußeren Bereiche
des Kristalls trotz leichter Streckung der Gitterstruktur, mit der die Oberfläche eher einem bcc-
statt einer hcp-Gitter glich, als Substrat prinzipiell geeignet. In den STM-Bildern waren ebene
Terrassen mit etwa \SI{35}{nm} Breite zu sehen, auf denen ein Wachstum von Gold gut zu
beobachten sein sollte.
\\
In verschiedenen Schichtdicken aufgdampftes Gold wurde ebenfalls mit LEED und STM untersucht. Es
wurde ein Stransky-Krastanov-Wachstum, d.h. ein Inselwachstum beobachtet. In den LEED-Aufnahmen
ergab dies bis zu den Schichtdicken von etwa zehn Monolagen diffuse Beugungsreflexe, was auf eine
wenig ausgeprägte Periodizität der Oberfläche schließen lässt. Ab zehn Monolagen wurden die
LEED-Reflexe ausgeprägter und das Untergrundleuchten schwächer, es ergab sich eine hexagonale
Periodizität der Goldoberfläche. Als mögliches Substrat für organische Moleküle, bei der große,
flache Inseln des Substrats zur Ermittlung der Anordnung und Struktur der Moleküle von Vorteil sind,
wurden von daher 20 und 30 Monolagen betrachtet. Da sich das Inselwachstum beider Schichtdicken was Ebenmäßgkeit und Größe
der Inseln anging - etwa \SI{35}{nm} Durchmesser der größten, 
\SI{7}{nm} Durchmesser der kleinsten, obersten Inseln - nicht signifikant voneinander
unterschied, wurden danach 20 Monolagen noch einmal bei etwa \SI{650}{K} getempert und mit den ungetemperten 20 Monolagen verglichen. Es
ergab sich eine Glättung der Oberfläche, d.h. die Inseln wurden größer mit einem Inseldurchmesser
von etwa \SI{15}{nm} bis \SI{35}{nm}. Verschiedene Höhenprofile entlang der Inseln ergaben ebenso,
dass die Inseln sehr eben sind.
\\
Somit würden 20 Monolagen auf dem Rheniumsubstrat bei etwa \SI{650}{K} getempert ausreichen, um
genügend große Inseln zu erhalten, auf denen organische Moleküle aufgebracht und untersucht werden
können. Eventuell könnte man durch Tempern bei höheren Temperaturen oder über einen längeren
Zeitraum eine weitere Glättung der Oberfläche erzielen.
\\

% SB:
% -Glasröhrchen in ventil, um mehr flüssigkeit reintun zu können (wenn über langen zeitraum viel riengemacht
% werden muss), spritzschutz
% -ausrichtung substrat in molekülkammer - sichtfenster weg! markierung an magnet?
% -konus oben spitzer machen? damit keine tropfenbildung auf "`terasse"'? oder konus oben vergrößern?
% -filtern der lösung oder verdünnen? geht filtern auch mit größeren molekülen?
% -zum säubern muss ventil abgeschraubt werden, da nicht erreichbares reservoir direkt über öffnung
% -evtl weg verkürzen mit kürzerem zwischenstück
% -weitere test mit quarzwaage zur skalierung? bzw groben abschätzung, da jedes molekül anderes
% gewicht

Der Test der Spraydepositionsapparatur erbrachte positive Ergebnisse, obgleich noch einige
Optimierungen des Aufbaus in Angriff genommen werden sollten. Der Aufbau wurde an der Testapparatur
gestestet, bei der als Substrat ein Glasträger zum Einsatz kam. Auf diesem sollten die verwendeten
Moleküle, blaues Kupfer-2-phthalocyanin und gelbes Tetracyanoquinodimethan, mit dem bloßen Auge gut
zu sehen sein. Trotz verschiedener getesteter Parameter konnte jedoch mit dem Originalaufbau keine
Farbe auf dem Träger beobachtet werden.
Zuletzt konnte jedoch mit einer Quarzwaage anstelle des
Glasträgers gezeigt werden, dass bei einer On-Time von \SI{3}{ms} und einer Off-Time von
\SI{500}{ms} genügend Moleküle auf der Waage ankommen, um die Frequenz des Schwingquarzes zu
ändern. Daher sollten auch bei einem Aufbau auf der Molekülkammer der UHV-Apparatur auf diese Weise
aufgebrachte Moleküle auf dem Substrat zum Liegen kommen. Interessant wäre festzustellen, wieviele
Monolagen bei wieviel Skalenteilen des Schwingquarzes auf dem Quarz bzw. auf dem Substrat haften
bleiben. Dies hängt natürlich von der Masse der verwendeten Moküle ab, sodass eine Kalibration des
Quarzes nur für das jeweilige aufgebrachte Molekül gilt. Andererseits kann eventuell über einen
Massenvergleich zwischen zur Kalibration benutztem Molekül und unbekanntem Molekül die Schichtdicke
pro Skalenteile abgeschätzt werden.
\\
Nicht zu vergessen ist hier natürlich das ebenfalls aufgesprühte Lösungsmittel, das im
Allgemeinen mit der Zeit auch wieder verdampft. Da das Lösungsmittel in der Regel nicht
mituntersucht werden soll, ist ebenfall interessant herauszufinden, wie lange man nach dem
Aufbringen der Moleküle etwa abwarten muss, bis das Lösungsmittel vom Substrat
verdampft ist.
\\
Ein wesentlicher Punkt beim Betrieb der Spraydeposition ist die Verstopfungsgefahr. Ohne den Aufbau
selbst zu verändern, könnte man diese verringern, indem man stark verdünnte Lösungen nutzt, am
besten mit Lösungsmitteln, in denen sich die jeweiligen Moleküle auch gut lösen. Das verwendete CuPc
ergab mit den verwendeten Lösungsmitteln Dimethylsulfoxid und Dimethylformamid nur eine grobe
Suspension. Die ungelösten groben Partikel setzten sich sehr schnell im Ventil bzw. dessem Reservoir
ab sowie auf der Spitze des Konus, was schnell zu den besagten Verstopfungen führte. Mit stark
verdünnten Lösungen könnte man diese Ablagerungen eventuell vermeiden oder zumindest hinauszögern.
Möglicherweise könnte man auch die Öffnung des Konus geringfügig vergrößern, allerdings wird dies
einen besseren Druckausgleich zwischen Würfel und unterem Aufbau der Apparatur ermöglichen. Ob
trotzdem noch ein ausreichendes Druckgefälle aufrecht erhalten werden kann, müsste getestet werden.
\\
Da das Reservoir des elektrischen Ventils nur ein sehr begrenztes Fassungsvermögen besitzt und man
gerade bei stark verdünnten Lösungen umso mehr der Lösung in die Apparatur bringen muss, um die
gleiche Anzahl an Molekülen auf dem Substrat zu deponieren, sollte das Reservoir erweitert werden.
Dies könnte man einfach realisieren, indem man ein schmales, hohes Glasröhrchen in das Reservoir
einführt und es an dessen Wänden so abdichtet, dass sich die Flüssigkeit dort nicht herausdrückt.
Ansonsten muss man, wie auch schon bei den Testreihen in dieser Arbeit geschehen, das Reservoir
während einer Messnung des öfteren Nachfüllen. Geschieht dies nicht rechtzeitig, kommt statt
der Lösung Luft in die Kammer, was die Probe kontaminieren wird.
\\
Leider konnte nicht, wie ursprünglich geplant, der Aufbau nach dem Test noch auf die Molekülkammer
der UHV-Apparatur aufgesetzt und die auf diese Weise auf den mit Gold präparierten Rheniumkristall
aufgebrachten Moleküle mit Hilfe der Rastertunnelmikroskopie untersucht werden. Es bleibt
herauszufinden, ob der Druck in der Molekülkammer erhalten werden kann, ob eine gleichmäßige
Verteilung der Moleküle auf dem Substrat stattfindet und ebenfalls, ob das ursprüngliche Ziel erreicht wird und die aufgebrachten Moleküle intakt bleiben.









