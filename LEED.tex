Eine einfache und verbreitete Methode, um etwas über Oberflächenstrukturen von
 Kristallen herauszufinden, ist eine Untersuchung mit LEED (Low Energy Electron
 Diffraction).
 Dabei wird ein Elektronenstrahl mit Energien von 20-300eV senkrecht auf die
 Oberfläche gerichtet, aus dem resultierende Beugungsmuster der elastisch
 gestreuten Elektronen können Rückschlüsse auf die Periodizität des Gitters
 gezogen werden.\\
 Abbildung bla zeigt den prinzipiellen Versuchsaufbau. Die Elektronenkanone
 besteht aus einem Thorium-beschichteten Iridium-Filament, das beim
 Aufheizen Elektronen emittiert. Mit Hilfe eines Wehneltzylinders kann die
 Intensität und die Fokussierung des Elektronenstrahls reguliert werden. Vor dem
 halbkugelförmigen fluoreszierenden Schirm befinden sich vier rebenso geformte
 Gitter, durch die inelastisch gestreute Elektronen mit wesentlich niedrigeren Energien
 davon abgehalten werden, auf dem Schirm störendes Hintergrundleuchten zu
 verursachen. Der Schirm selbst ist positiv geladen, um die höherenergetischen
 elastisch gestreuten Elektronen auf dem letzten Weg zum Schirm zu
 beschleunigen.\\
  Da die mittlere freie Weglänge solch niedrigenergetischer Elektronen in
 Festkörpern sehr gering ist, dringen sie nur in die obersten schichten des
 Kristalls ein. Damit ist diese Untersuchungsmethode sehr oberflächensensitiv.
 Die Wellenlänge der Elektronen liegt im Bereich des Gitterabstandes und
 berechnet sich über die deBroglie Beziehung:
 \[\lambda=\frac{h}{\sqrt{2mE}}\approx \sqrt{\frac{1{,}5}{E_{eV}}}\]
Um das Beugungsmuster zu verstehen, betrachtet man vereinfacht eine elastische
Streuung nur an der obersten Atomschicht. Im Falle der elastischen Streuung ist
der Betrag des ausfallenden Wellenvektors $\vec{k'}$ gleich dem des
einfallenden Vektors $\vec{k}$, also $k=k'$. Für konstruktive Interferenz
muss die Laue Bedingung erfüllt sein:
\[\vec{k}-\vec{k'}=\vec{G}\]
mit dem reziproken Gittervektor $\vec{G}$. 
In diesem Fall der Oberflächenstreuung gibt es keine Beugungsbedingungen
senkrecht zur Probenoberfläche, sodass wir konstruktive Interferenz erhalten,
wenn die 
