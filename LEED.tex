\FloatBarrier

Eine einfache und verbreitete Methode, um etwas über Oberflächenstrukturen von
 Kristallen herauszufinden, ist eine Untersuchung mit LEED (Low Energy Electron
 Diffraction).
 Dabei wird ein Elektronenstrahl mit Energien von 20-400eV senkrecht auf die
 Oberfläche gerichtet. Aus dem resultierenden
 Beugungsmuster der elastisch gestreuten Elektronen können Rückschlüsse auf die Periodizität des Gitters
 gezogen werden.


 \begin{figure}[H]
	\centering
	\sffamily 
	\includesvg{leed}
	\caption{\textit{Schema des LEED-Aufbaus. Die am Filament emittierten Elektronen werden an der
	Probenoberfläche gebeugt. Inelastisch gestreute Elektronen mit niedrigerer Energie werden durch die
	Gitter abgebremst, die elastisch gestreuten Elektronen dagegen werden zum Leuchtschirm
	hin beschleunigt. Dort kann dann das Beugungsbild beobachtet werden. In der Skizze sind nur zwei
	Gitter angedeutet, tatsächlich sind im verwendeten Aufbau vier Gitter verbaut.}}
	\label{leedaufbau}
\end{figure}




 Abbildung \ref{leedaufbau} zeigt den prinzipiellen Versuchsaufbau einer LEED-Untersuchung. Die
 Elektronenkanone besteht aus einem Thorium-beschichteten Iridium-Filament, das beim Aufheizen
 Elektronen emittiert. Mit Hilfe eines Wehneltzylinders kann die Intensität und die Fokussierung
 des Elektronenstrahls reguliert werden. Vor dem halbkugelförmigen fluoreszierenden Schirm befinden
 sich vier ebenso geformte Gitter, durch die inelastisch gestreute Elektronen mit wesentlich
 niedrigeren Energien davon abgehalten werden, auf dem Schirm störendes Hintergrundleuchten zu
 verursachen. Der Schirm selbst ist positiv geladen, um die höherenergetischen elastisch gestreuten
 Elektronen auf dem letzten Weg zum Schirm zu beschleunigen.\\
  Da die mittlere freie Weglänge solch niedrigenergetischer Elektronen in
 Festkörpern sehr gering ist, dringen sie nur in die obersten Schichten des
 Kristalls ein. Damit ist diese Untersuchungsmethode sehr oberflächensensitiv.
 Die Wellenlänge der Elektronen mit Energie $E$ liegt im Bereich des Gitterabstandes und
 berechnet sich über die deBroglie Beziehung:
 \[\lambda=\frac{h}{\sqrt{2mE}}\approx \sqrt{\frac{1{,}5}{E_{eV}}}\]
Um das Beugungsmuster zu verstehen, betrachtet man vereinfacht eine elastische
Streuung nur an der obersten Atomschicht. Im Falle der elastischen Streuung ist
der Betrag des ausfallenden Wellenvektors $\vec{k'}$ gleich dem des
einfallenden Vektors $\vec{k}$, also $k=k'$. Für konstruktive Interferenz
muss die Laue Bedingung erfüllt sein:
\[\vec{k}-\vec{k'}=\Delta \vec{k}=\vec{G}\]
mit dem reziproken Gittervektor $\vec{G}$, genauso wie die sogenannten Laue-Gleichungen
\[\vec{a}_i\cdot \Delta \vec{k}=2\pi h_i,~~~~~~~i=1, 2, 3,\]

wobei $\vec{a}_i$ die primitiven Gittervektoren des Realraums und $h_i$ die Millerschen Indizes
sind. In diesem Fall der Oberflächenstreuung erhalten wir konstruktive Interferenz,
wenn die Vektorkomponten parallel zur Oberfläche, $\Delta\vec{k}_{||}=\vec{k}_{||}-\vec{k'}_{||}$,
einem reziproken Gittervektor der zweidimensionalen Oberfläche entspricht. Senkrecht zur Probenoberfläche
gibt es keine solche Beugungsbedingung. Möchte man nun die Ewald-Konstruktion auf dieses
zweidimensionale Problem anwenden, muss die dritte Laue-Bedingung fallen gelassen werden. Statt
diskreter Punkte in der dritten Dimension werden nun jedem Gitterpunkt ($h_1, h_2$) durchgehende
Linien senkrecht zur Oberfläche zugewiesen. Veranschaulicht ist diese Konstruktion in Abb.
\ref{ewald}. Der einkommende Strahl mit Wellenvektor $\vec{k}$ wird mit seiner Spitze auf einem
reziproken Gitterpunkt positioniert, sein Anfang dient als Mittelpunkt der Ewaldkugel. Die
Bedingung $\Delta \vec{k}_{||}=\vec{G}_{||}$ ist nun für jeden Punkt erfüllt, an dem die Kugel eine
der Linien schneidet. 
\begin{figure}[H]
\centering
\sffamily 
\includesvg{ewald}
\caption{\textit{Die Konstruktion der Ewaldkugel einer elastischen Streuung an einer
zweidimensionalen Oberfläche. Die reziproken Gitterpunkte} (hk) \textit{sind auf einer Achse $k_x$
eingezeichnet. Die Bedingung für konstruktive Interferenz, $\Delta \vec{k}_{||}=\vec{G}_{||}$,  ist erfüllt, wo die Kugel eine der senkrechten Linien
schneidet, z.B. bei $(30)$. }}
\label{ewald}
\end{figure}




Bei einer realen Streuung muss natürlich auch die Streuung der Elektronen an tieferen
Schichten berücksichtigt werden, sodass die dritte Laue-Bedingung nicht komplett fallen gelassen werden
kann. Je tiefer die Elektronen in den Kristall eindringen, desto mehr Beugungseffekte senkrecht zur
Oberfläche tragen zum Beugungsbild ein, desto wichtiger wird also die Laue-Gleichung für diese
Dimension. Nicht zu vernachlässigen sind außerdem einerseits mehrfach gestreute Elektronen sowie
inelastisch gestreute Elektronen, die in dieser simplen Konstruktion nicht mit einbezogen werden. Im
Aufbau können diese jedoch durch die Linsen größtenteils herausgefiltert werden. Näheres zu einer dynamischen
Theorie der Elektronenbeugung in LEED findet sich in \cite{Lueth}.

\FloatBarrier










