
Seit der Entdeckung elektrischer Leitfähigkeit in diversen organischen Materialien wird an deren
Einsatz als organische Halbleitermaterialien in elektronischen Schaltungen geforscht. Mittlerweile
konnten u.a. mit organischen Transistoren, LEDs und Photovoltaikzellen große Erfolge in dieser
Richtung gefeiert werden. Das Interesse an diesen Materialien ist aufgrund der niedrigen
Herstellungskosten und der Möglichkeit, leichte und flexible elektronische Bauteile für
beispielsweise biegsame Displays herzustellen, ungebrochen. 
\\
Die Herstellung solcher Geräte bzw. Bauteile erfordern dünne Schichten organischer Moleküle, in
denen die Moleküle strukturiert angeordnet sind. Zu verstehen, wie und auf welchem Untergrund solche
Schichten wachsen können und wie man sie manipulieren kann, ist also von großer Bedeutung für die
Anfertigung effizienter organischer Elektronik.
\\
In dieser Arbeit wurden zwei Ziele verfolgt. Zum einen wurde das Wachstum dünner Goldschichten auf
Rhenium untersucht. Gold ist als Substrat für organische Moleküle sehr interessant, da es inert ist
und somit zum Beispiel im Ultrahochvakuum, in dem die Versuche stattfanden, nur wenig mit den
Restgasmolekülen reagiert. Dennoch werden organische Moleküle gut auf Goldoberflächen adsorbiert. 
Rhenium ist ein seltenes Metall mit dem Elementsymbol Re und der Ordnungszahl 75. Es hat einen hohen
Schmelzpunkt von $3186\celsius$, der nur noch von Wolfram und Kohlenstoff übertroffen wird. Rhenium
hat den Vorteil gegenüber anderen Metallen wie Wolfram und Molybdän, dass es auch nach dem
Verarbeiten wie Schweißen oder Schmieden duktil bleibt und nicht spröde wird.
\\
Das Wachstum weniger Schichten Gold auf einem Rheniumkristall in (0001)-Orientierung wurde mit LEED
(Low Energy Electron Diffraction) und Rastertunnelmikroskopie untersucht, zwei sehr
oberflächensensitive Untersuchungenmethoden, mit der sich Periodizität der Oberfläche,
Oberflächengüte und die Art des Wachstums gut dokumentieren lassen.
\\
Im zweiten Teil der Arbeit wurde eine Apparatur zum Aufbringen von organischen Molekülen auf
Substraten wie dem Re-Kristall durch Sprühbeschichtungs- bzw. Spraydepositionsverfahren gebaut und
getestet. In der Regel kommt hierfür die Molekularstrahlepitaxie (MBE) zum Einsatz, bei der die
Moleküle in einem Tiegel erhitzt werden und der entstehende Dampf auf der Substratoberfläche kondensiert.
Dabei kann es allerdings passieren, dass durch die thermische Energie Molekülbindungen aufbrechen
und das Molekül seine Form verliert bzw. "`zerbricht"'. Dies soll durch die Spraydeposition
verhindert werden. Statt die Moleküle zu verdampfen, werden sie über Druckunterschiede auf das
Substrat aufgebracht, zum Beispiel über Airbrushpistolen oder, wie in dieser Arbeit, über das
Einlassen der in einer Lösung befindlichen Moleküle in eine Vakuumkammer über ein Ventil.
\\
In einigen Bereich wird das Spraydepositionsverfahren schon erfolgreich getestet, zum Beispiel für
organische Solarzellen \cite{Tait}\cite{Hoth}, Leuchtdioden \cite{Wu} oder Photodioden \cite{Ted}.
Es bietet eine Möglichkeit, organische Moleküle großflächig und kostengünstig auf Substrate aufzubringen
\cite{Sun}\cite{Seok} und ist daher eine ernstzunehmende Alternative für bisherige Verfahren wie
die MBE oder Rotationsbeschichtung (Spin Coating) \cite{Alaa}.
