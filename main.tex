%_______________________________________________________________________________
%class
%_______________________________________________________________________________
%\documentclass[a4paper,11pt,onecolumn,final,german,openbib]{scrbook}
\documentclass[a4paper,11pt,oneside,final,german,openbib,pdftex]{scrbook}
%_______________________________________________________________________________
% page borders
%_______________________________________________________________________________
\addtolength{\headheight}{2cm}
%\addtolength{\topmargin}{2cm}
\setlength{\oddsidemargin}{1.0cm}
\setlength{\evensidemargin}{0.5cm}
\setlength{\textwidth}{14.3cm}
\setlength{\parindent}{0mm}

% Default folder for graphics
 
%_______________________________________________________________________________
% packages
%_______________________________________________________________________________
\usepackage{german}
\usepackage{amsmath, amssymb}
\usepackage[utf8]{inputenc}
\usepackage{graphicx}
\usepackage{enumerate}
\usepackage{multirow}
\usepackage{subfigure}
\usepackage{dsfont}
\usepackage{slashed} 
\usepackage{textcomp}
\usepackage{svg}
\usepackage{pdfpages}
\usepackage{placeins}
\usepackage{float}
\usepackage[labelfont=bf, format=plain]{caption}
\usepackage{cite}
\usepackage[decimalsymbol=comma]{siunitx}
\usepackage{setspace}

\usepackage[percent]{overpic}


%_______________________________________________________________________________
% bold fonts for headings
%_______________________________________________________________________________
\font\afont=cmssbx10 scaled \magstep5     % for the title
\font\bfont=cmssbx10 scaled \magstep4     % for chapter headings
\font\cfont=cmssbx10 scaled \magstep3
\font\dfont=cmssbx10 scaled \magstep2     % for section headings and author name
\font\efont=cmssbx10 scaled \magstephalf

%_______________________________________________________________________________
% index depth
%_______________________________________________________________________________
\setcounter{secnumdepth}{3}
\setcounter{tocdepth}{3}

%_______________________________________________________________________________
% new commands
%_______________________________________________________________________________
\newcommand{\demi}{\frac{1}{2}}

%_______________________________________________________________________________
% renewed commands
%_______________________________________________________________________________
% \renewcommand{\topfraction}{1.}       % this is important for figure placement
% \renewcommand{\bottomfraction}{1.}
\makeatletter
\renewcommand\paragraph{\@startsection{paragraph}{4}{\z@}%
  {-3.25ex\@plus -1ex \@minus -.2ex}%
  {1.5ex \@plus .2ex}%
  {\normalfont\normalsize\bfseries}
}
\makeatother


%_______________________________________________________________________________
% special words, hyphenation
%_______________________________________________________________________________
\hyphenation{Ba-che-lor-ar-beit}

\pagestyle{empty}
\pagestyle{headings}
%for changing the style on a specific page use \thispagestyle{e.g., empty}

%_______________________________________________________________________________
%_______________________________________________________________________________
\begin{document}
\pagenumbering{roman}

% _______________________________________________________________________________
\begin{titlepage}
  \vspace*{6mm}
  \begin{center}
  \linespread{1.3}
     {\sffamily \bfseries \Huge  %\afont
     Untersuchung des Wachstums von Au auf Re(111) mittels LEED und STM
     sowie Aufbau und Test einer Sprühdepositionsapparatur\par}
     \vspace*{3.5cm}
     {\large von}
     \\[3.5cm]
     \linespread{1}
     {\dfont Verena Grimm}
     \\[2cm]
     {\large Bachelorarbeit in Physik \/\\
        vorgelegt dem Fachbereich Physik, Mathematik und Informatik (FB 08) \/\\
        der Johannes Gutenberg-Universit\"at Mainz \/\\
        am 03. Juni 2014}
   \end{center}
   \vfill
   1. Gutachter: Prof. Dr. Hans-Joachim Elmers\\
   2. Gutachter: Dr. Martin Jourdan\\
   \vfill
\end{titlepage}



\thispagestyle{empty}
Ich versichere, dass ich die Arbeit selbstst\"andig verfasst und keine
anderen als die angegebenen Quellen und Hilfsmittel benutzt sowie
Zitate kenntlich gemacht habe.
\\
\\[3.5cm]
Mainz, den [Datum] [Unterschrift]
\vfill
\noindent
Verena Grimm\\
KOMET\\
Institut f\"ur Physik\\
Staudingerweg 7\\
Johannes Gutenberg-Universit\"at
D-55128 Mainz\\
{\tt vegrimm@students.uni-mainz.de}

%_______________________________________________________________________________
\renewcommand\contentsname{Inhaltsverzeichnis}
\renewcommand\figurename{Abbildung}
\renewcommand\tablename{Tabelle}
\tableofcontents
\clearpage 

\mainmatter  
\sloppy

%_______________________________________________________________________________
% \chapter{Einleitung}





%_______________________________________________________________________________
\chapter{Grundlagen}


% \subsection{Oberflächenstrukturen}
% \import{}{oberflaechen}
\section{Rastertunnelmikroskopie}
\import{theorie/}{starting_stm.tex}
 \section{LEED}
 \import{theorie/}{LEED.tex}


 \chapter{Versuchsaufbau} 
 \import{versuchsaufbau/}{starting_versuchsaufbau}
 
\chapter{Wachstum dünner Au-Schichten auf Re(111)}
\import{goldaufre/}{goldaufre}
 
\chapter{Test der Spraydepositionsapparatur} \label{kaptest}
\import{testsb/}{starting_test_sb}



% 
% \chapter{Zusammenfassung und Ausblick}
% Gold auf Re:





SB:
-Glasröhrchen in ventil, um mehr flüssigkeit reintun zu können (wenn über langen zeitraum viel riengemacht
werden muss), spritzschutz
-ausrichtung substrat in molekülkammer - sichtfenster weg! markierung an magnet?
-konus oben spitzer machen? damit keine tropfenbildung auf "`terasse"'? oder konus oben vergrößern?
-filtern der lösung oder verdünnen? geht filtern auch mit größeren molekülen?
-zum säubern muss ventil abgeschraubt werden, da nicht erreichbares reservoir direkt über öffnung
-evtl weg verkürzen mit kürzerem zwischenstück
-weitere test mit quarzwaage zur skalierung? bzw groben abschätzung, da jedes molekül anderes
gewicht



%_______________________________________________________________________________
% \begin{appendix}
% \chapter{Anhang}
% 
% \section{Tabellen und Abbildungen}
% 
% 
% 
% %_______________________________________________________________________________
% \section{Weiterf\"uhrende Details zur Arbeit}



%_______________________________________________________________________________


\bibliography{literatur}{}
\bibliographystyle{alpha}



%_______________________________________________________________________________
% \chapter{Danksagung}
% 
% ... an wen auch immer. Denken Sie an Ihre Freundinnen und Freunde, 
% Familie, Lehrer, Berater und Kollegen.
% 
% \end{appendix}

\end{document}  
        
        