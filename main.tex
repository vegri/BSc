%_______________________________________________________________________________
%class
%_______________________________________________________________________________
%\documentclass[a4paper,11pt,onecolumn,final,german,openbib]{scrbook}
\documentclass[a4paper,11pt,oneside,final,german,openbib,pdftex]{scrbook}
%_______________________________________________________________________________
% page borders
%_______________________________________________________________________________
\addtolength{\headheight}{2cm}
%\addtolength{\topmargin}{2cm}
\setlength{\oddsidemargin}{1.0cm}
\setlength{\evensidemargin}{0.5cm}
\setlength{\textwidth}{14.3cm}
\setlength{\parindent}{0mm}
 
%_______________________________________________________________________________
% packages
%_______________________________________________________________________________
\usepackage{german}
\usepackage{amsmath, amssymb}
\usepackage[utf8]{inputenc}
\usepackage{graphicx}
\usepackage{enumerate}
\usepackage{multirow}
\usepackage{subfigure}
\usepackage{dsfont}
\usepackage{slashed}
\usepackage{textcomp}
%\usepackage{url}

%_______________________________________________________________________________
% bold fonts for headings
%_______________________________________________________________________________
\font\afont=cmssbx10 scaled \magstep5     % for the title
\font\bfont=cmssbx10 scaled \magstep4     % for chapter headings
\font\cfont=cmssbx10 scaled \magstep3
\font\dfont=cmssbx10 scaled \magstep2     % for section headings and author name
\font\efont=cmssbx10 scaled \magstephalf

%_______________________________________________________________________________
% index depth
%_______________________________________________________________________________
\setcounter{secnumdepth}{3}
\setcounter{tocdepth}{3}

%_______________________________________________________________________________
% new commands
%_______________________________________________________________________________
\newcommand{\demi}{\frac{1}{2}}

%_______________________________________________________________________________
% renewed commands
%_______________________________________________________________________________
% \renewcommand{\topfraction}{1.}       % this is important for figure placement
% \renewcommand{\bottomfraction}{1.}
\makeatletter
\renewcommand\paragraph{\@startsection{paragraph}{4}{\z@}%
  {-3.25ex\@plus -1ex \@minus -.2ex}%
  {1.5ex \@plus .2ex}%
  {\normalfont\normalsize\bfseries}
}
\makeatother

%_______________________________________________________________________________
% special words, hyphenation
%_______________________________________________________________________________
\hyphenation{Ba-che-lor-ar-beit}

\pagestyle{empty}
\pagestyle{headings}
%for changing the style on a specific page use \thispagestyle{e.g., empty}

%_______________________________________________________________________________
%_______________________________________________________________________________
\begin{document}
\pagenumbering{roman}

%_______________________________________________________________________________
\begin{titlepage}
  \vspace*{6mm}
  \begin{center}
     {\afont Rastertunnelmikroskopie an adsorbierenden Schichten}
     \\[3.5cm]
     {\large von}
     \\[3.5cm]
     {\dfont Verena Grimm}
     \\[2cm]
     {\large Bachelorarbeit in Physik \/\\
        vorgelegt dem Fachbereich Physik, Mathematik und Informatik (FB 08) \/\\
        der Johannes Gutenberg-Universit\"at Mainz \/\\
        am 66. Juni 2014}
   \end{center}
   \vfill
   1. Gutachter: Prof. Dr. Hans-Joachim Elmers\\	
   2. Gutachter: Prof. Dr. Un Bekannt \\
   \vfill
\end{titlepage}

\thispagestyle{empty}
Ich versichere, dass ich die Arbeit selbstst\"andig verfasst und keine 
anderen als die angegebenen Quellen und Hilfsmittel benutzt sowie 
Zitate kenntlich gemacht habe.
\\
\\[3.5cm] 
Mainz, den [Datum] [Unterschrift]
\vfill
\noindent 
Verena Grimm\\
KOMET\\
Institut f\"ur Physik\\
Staudingerweg 7\\
Johannes Gutenberg-Universit\"at
D-55128 Mainz\\
{\tt vegrimm@students.uni-mainz.de}

%_______________________________________________________________________________
\renewcommand\contentsname{Inhaltsverzeichnis}
\renewcommand\figurename{Abbildung}
\renewcommand\tablename{Tabelle}
\tableofcontents
\clearpage

\mainmatter
\sloppy

%_______________________________________________________________________________
\chapter{Einleitung}





%_______________________________________________________________________________
\chapter{Hauptteil}




%_______________________________________________________________________________
\section{Grundlagen}


\subsection{Rastertunnelmikroskopie/-spektroskopie}
\subsection{LEED-Untersuchung}
\subsection{noch was?}

\section{Versuchsaufbau}


\section{Aufbringen dünner Au-Schichten auf einem Re-Kristall}


\section{Bau eines Dinges}


\section{Test des Dinges: Aufbringen von Molekül auf Substrat}




\section{Ergebnisse}



%_______________________________________________________________________________
\chapter{Zusammenfassung und Ausblick}



%_______________________________________________________________________________
\begin{appendix}
\chapter{Anhang}

\section{Tabellen und Abbildungen}



%_______________________________________________________________________________
\section{Weiterf\"uhrende Details zur Arbeit}



%_______________________________________________________________________________
\chapter{Literaturverzeichnis}




%_______________________________________________________________________________
\chapter{Danksagung}

... an wen auch immer. Denken Sie an Ihre Freundinnen und Freunde, 
Familie, Lehrer, Berater und Kollegen.

\end{appendix}

\end{document}  
        
        