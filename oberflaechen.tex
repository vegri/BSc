Im folgenden werden die Oberflächenstrukturen von Festkörpern mit periodischem Aufbau näher
 beleuchtet. \\
 Betrachtet man die Oberfläche eines solchen Festkörpers, so ist leicht einsehbar, dass die
 Kräfte zwischen Atomen der äußersten Schicht durch fehlende Nachbarn auf einer Seite beträchtlich
 anders sind. Die Gleichgewichtsbedingungen der Oberflächenatome sind im Vergleich zum
 Körperinneren geändert, sodass man in der Regel andere Positionen der Atome und eine andere Oberflächenstruktur
 als im Inneren erwartet. Die Oberfläche ist in diesem Sinn nicht einfach wie ein (gedanklicher)
 Schnitt durch den Kristall.\\
 Verbreitete Veränderungen sind zum Beispiel die Relaxation und die Rekonstruktion. Bei der
 Relaxation bleibt die Periodizität des zweidimensionalen Gitters die Gleiche wie die der
 körperinneren Schichten, jedoch ist der Abstand dieser Schicht zu der
 darunterliegenden kleiner, seltener auch größer (siehe Abb. \ref{blabla}). Die Rekonstruktion
 beschreibt Verschiebungen parallel zur Oberfläche. Dabei kann sich die Größe der
 zweidimensionalen Einheitszelle verändern, aber auch deren Lage hinsichtlich der darunterliegenden
 Schichten (z.B. Drehungen) oder ihre Gestalt.  Beispielsweise rekonstruiert Gold die
 (100)-Oberfläche von einer fcc-Struktur zu einer hcp-Struktur. [Quelle?] Weiterhin können Atome
 oder ganze Reihen von Atomen auf der Oberfläche fehlen im Vergleich zum Inneren. 