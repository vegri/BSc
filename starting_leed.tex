Was weiß ich über leed?
Um etwas über die Struktur von (metallischen) Oberflächen zu erfahren, ist eine Methode, langsame Elektronen
mit einer Energie von 100-150eV in einem senkrechten Strahl auf die Oberfläche zu schicken und sich das
Beugungsmuster zu betrachten, das eine Abbildung des reziproken Gitters an der Oberfläche darstellt. Die
Methode einer solchen Beugung niederenergetischer Elektronen nennt man LEED (Low Energy Electron Diffraction).
 

 Eine einfache und verbreitete Methode, um etwas über Oberflächenstrukturen von
 Kristallen herauszufinden, ist eine Untersuchung mit LEED (Low Energy Electron
 Diffraction).
 Dabei wird ein Elektronenstrahl mit Energien von 20-300eV senkrecht auf die
 Oberfläche gerichtet, aus dem resultierende Beugungsmuster der elastisch
 gestreuten Elektronen können Rückschlüsse auf die Periodizität des Gitters
 gezogen werden.\\
 Abbildung bla zeigt den prinzipiellen Versuchsaufbau. Die Elektronenkanone
 besteht aus einem heißen Thorium-beschichteten Iridium-Filament, das beim
 Aufheizen Elektronen emittiert. Mit Hilfe eines Wehneltzylinders kann die
 Intensität und die Fokussierung des Elektronenstrahls reguliert werden. Vor dem
 halbkugelförmigen fluoreszierenden Schirm befinden sich vie rebenso geformte
 Gitter, durch die inelastisch gestreute Elektronen mit wesentlich niedrigeren Energien
 davon abgehalten werden, auf dem Schirm störendes Hintergrundleuchten zu
 verursachen. Der Schirm selbst ist positiv geladen, um die höherenergetischen
 elastisch gestreuten Elektronen auf dem letzten Weg zum Schirm zu
 beschleunigen.\\
  Da die mittlere freie Weglänge solch niedrigenergetischer Elektronen in
 Festkörpern sehr gering ist, dringen sie nur in die obersten schichten des
 Kristalls ein. Damit ist diese Untersuchungsmethode sehr oberflächensensitiv.
 Die Wellenlänge der Elektronen liegt im Bereich des Gitterabstandes und
 berechnet sich über die deBroglie Beziehung:
 \[\lambda=\frac{h}{\sqrt{2mE}\approx \sqrt{\frac{1{,}5}{E_{eV}}}}\]
Um das Beugungsmuster zu verstehen, betrachtet man vereinfacht eine elastische
Streuung nur an der obersten Atomschicht. Im Falle der elastischen Streuung ist
der Betrag des ausfallenden Wellenvektors gleich dem des einfallenden,
$|k|=|k'|$
 
 

 
 

 Stichpunkte:
 (extra punkt zu oberflächenstrukturen?)
 - wegen fehlender nachbaratome der oberflächenatome auf einer seite, kräfte zwischen atomen in äußester
 schicht sind beträchlich geändert
 - gleichgewichtsbedingungen der oberflächenatome sind im vergleich zum körperinneren geändert
 - deswegen erwartet man andere positionen der atome und eine andere oberflächenstruktur die in der regel
 nicht mit der des körperinneren übereinstimmt
 - heißt: oberfläche ist nicht einfach ein schnitt/abbruch des kristalls
 - beispiele: relaxation und rekonstruktion (was ist das)
 - rekonsruktionen meinstens über mehr als eine atomlage in den kristall hinein 
 - defekte
 
 LEED
 -  auch langsame elektronen haben eine nicht zu vernachlässigende eindringtiefe von mindestens einigen
 monolagen; trotzdem ist die oberste lage vorherrschend bei den oberflächenuntersuchungen
 - um oberflächen zu beschreiben, kann man dennoch (bei geringer defektdichte) von einem perfekt periodischen
 zweidimensionalen gitter ausgehen
 - 
 
 