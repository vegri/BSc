Was weiß ich über leed?
Um etwas über die Struktur von (metallischen) Oberflächen zu erfahren, ist eine Methode, langsame Elektronen
mit einer Energie von 100-150eV in einem senkrechten Strahl auf die Oberfläche zu schicken und sich das
Beugungsmuster zu betrachten, das eine Abbildung des reziproken Gitters an der Oberfläche darstellt. Die
Methode einer solchen Beugung niederenergetischer Elektronen nennt man LEED (Low Energy Electron Diffraction).
 

 Stichpunkte:
 (extra punkt zu oberflächenstrukturen?)
 - wegen fehlender nachbaratome der oberflächenatome auf einer seite, kräfte zwischen atomen in äußester
 schicht sind beträchlich geändert
 - gleichgewichtsbedingungen der oberflächenatome sind im vergleich zum körperinneren geändert
 - deswegen erwartet man andere positionen der atome und eine andere oberflächenstruktur die in der regel
 nicht mit der des körperinneren übereinstimmt
 - heißt: oberfläche ist nicht einfach ein schnitt/abbruch des kristalls
 - beispiele: relaxation und rekonstruktion (was ist das)
 - rekonsruktionen meinstens über mehr als eine atomlage in den kristall hinein 
 - defekte
 
 LEED
 -  auch langsame elektronen haben eine nicht zu vernachlässigende eindringtiefe von mindestens einigen
 monolagen; trotzdem ist die oberste lage vorherrschend bei den oberflächenuntersuchungen
 - um oberflächen zu beschreiben, kann man dennoch (bei geringer defektdichte) von einem perfekt periodischen
 zweidimensionalen gitter ausgehen
 - 
 
 