- wegen fehlender nachbaratome der oberflächenatome auf einer seite, kräfte zwischen atomen in äußester
 schicht sind beträchlich geändert
 - gleichgewichtsbedingungen der oberflächenatome sind im vergleich zum körperinneren geändert
 - deswegen erwartet man andere positionen der atome und eine andere oberflächenstruktur die in der regel
 nicht mit der des körperinneren übereinstimmt
 - heißt: oberfläche ist nicht einfach ein schnitt/abbruch des kristalls
 - beispiele: relaxation und rekonstruktion (was ist das)
 - rekonsruktionen meinstens über mehr als eine atomlage in den kristall hinein 
 - defekte
 
 Im folgenden werden die Oberflächenstrukturen von Festkörpern mit periodischem Aufbau näher
 beleuchtet. \\
 Betrachtet man die Oberfläche eines solchen Festkörpers, so ist leicht einsehbar, dass sich die
 Kräfte zwischen Atomen der äußersten Schicht durch fehlende Nachbarn auf einer Seite beträchtlich
 ändern. die Gleichgewichtsbedingungen der Oberflächenatome sind im Vergleich zum Körperinneren
 geändert, sodass man in der Regel andere Positionen der Atome und eine andere Oberflächenstruktur
 als im Inneren erwartet, sodass die Oberfläche nicht einfach wie ein (gedanklicher) Schnitt durch
 den Kristall ist.\\
 
