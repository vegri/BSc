Das Rastertunnelmikroskop (engl. Scanning Tunneling Microscope, kurz: STM) ist eine Methode der
Rastersondenmikroskopie, bei der eine dreidimensionale Aufnahme atomarer Oberflächen elektrisch
leitender Festkörper erstellt werden kann. Die Funktionsweise beruht auf dem quantenmechanischen
Tunneleffekt (siehe Abb. \ref{tunnel}): Wird eine sehr feine, leitende Spitze - idealerweise nur ein
Atom - bis auf wenige {\AA} an die Oberfläche heran gebracht und eine Spannung angelegt, fließt zwischen Spitze und
Oberfläche ein Strom. Durch die kurze Distanz ist die Tunnelbarriere, die das Vakuum darstellt,
endlich, und je nach Polung der Spannung können die Elektronen von der Spitze zur Oberfläche tunneln bzw.
umgekehrt. Der Strom ist dabei abhängig von der elektronischen Zustandsdichte der Spitze und der
Probe an dieser Stelle sowie vom Abstand. Wird die Spitze über die Oberfläche bewegt und Strom bzw.
Abstand gemessen, kann auf diese Weise ein topographisches Bild der Oberfläche mit atomarer
Auflösung erreicht werden.\\
\begin{figure}[H]
\captionsetup{format=plain}
\centering
\includegraphics[width=5cm]{TunnelEffekt.png}
\caption{\textit{Veranschaulichung des Tunneleffekts. Ein Elektron mit Energie $E$ kann nach
klassischen Gesetzen die Potentialbarriere $V$ mit $V>E$ nicht überwinden, tatsächlich besteht
jedoch nach quantenmechanischer Rechnung eine endliche Wahrscheinlichkeit dafür, das Teilchen
außerhalb der Barriere zu finden. Das Elektron durchtunnelt die Barriere, wobei die
Aufenthaltswahrscheinlichkeit $|\Psi|²$ innerhalb der Barriere exponentiell abklingt. }}
\label{stmaufbau}
\end{figure}

 
ok was weiß ich?
- schwer zu beschreiben, weil funktionen spitze und oberfläche unbekannt
- näherungen nach tersoff und hamann, die einiges vereinfachen, und zwar\ldots
- damit kann man ne aussage über den strom machen, der dann folgende gestalt annimmt: \ldots
- das mit der auflösung passt noch nicht so ganz, aber das kommt daher, weil die nur s orbitale
betrachtet haben, was besser wird wenn man d und p orbitale mit dazunimmt, und außerdem das
effektive potential, was binnig vorgeschlagen hat
