Der Aufbau zur Spraydeposition wurde an einer kleineren HV-Apparatur getestet (siehe \ref{testapparatur}). Der
Vorteil ist, dass diese Kammer wegen ihres kleineren Volumens schneller abgepumpt werden kann, was bei
auftretenden Problemen, bei denen man die Kammer öffnen muss, ein schnelleres Arbeiten ermöglicht.\\
Die sogenannte Testapparatur besteht aus zwei Pumpen (Vorpumpe (1), Marke bla, und Hauptpumpe (2), Marke bla),
einer Hauptkammer mit Massenspektrometer (3) und Druckmessgerät bla (4) sowie dem aufgesetzten
Sprühbeschichter (4). Am Würfel des Beschichters ist eine (weitere) Membranpumpe angebracht (5). Das Ventil
zum Einspritzen des aufzutragenden Moleküls (6) wird angesteuert über ein Netzteil von IotaOne (7). Das
Substrat, hier zum Test ein Glasträger, befindet sich unterhalb des Beschichters (7.5) und kann durch einen
Fensterflansch beobachtet werden.\\
 Zu Beginn der Testreihe wurde das große Ventil (8) geschlossen und Beschichter sowie die Hauptkammer getrennt
 gepumpt, bis beide Vakuumpumpen vollständig hochgefahren waren. Um ein Druckgefälle zwischen Würfel und
 Hauptkammer herzustellen, wurde dann das kleine Ventil (9) geschlossen und das große Ventil geöffnet. Nun
 konnte das gelöste Molekül über eine Pipette in das obere Ventil (?) eingefüllt und dieses über das Netzteil
 über einen einstellbaren Zeitraum geöffnet werden.\\
Bei dem zum Testen benutzten Molekül handelte es sich um Kupfer-2-phthalocyanin (kurz CuPc), ein blauer
Feststoff, der als Pigment beispielsweise in Kunststoffen, Lacken oder Druckerfarben verwendet wird. Das CuPc
wurde in dem farblosen organischen Lösungsmittel Dimethylsulfoxid (kurz DMSO) aufgelöst, woraus sich eine
grobe Suspension ergab. Nachdem diese in das oberste Ventil hinein geträufelt wurde, mussten geeignete
Parameter zur Öffnung des Ventils gefunden werden. Die Ansteuerung ermöglicht prinzipiell zwei Modi: ein
einmaliges Öffnen des Ventils für einen einstellbaren Zeitraum (One Shot Modus) und einen Zyklus von
abwechselndem Öffnen und Schließen des Ventils (Cycle Modus). Da ein Einspritzen der Lösung in das Vakuum
natürlich den Druck steigen lässt, wurde hier (zunächst) der Cycle Modus gewählt, da so mutmaßlich der Druck
durch das Schließen des Ventils zwischendurch besser erhalten werden kann. In diesem Modus müssen die Zeiten
für das Öffnen (On Time) und des Schließens (Off Time) gewählt werden, wobei die Off Time
betriebsbedingt (?) mindestens zehnmal so lang wie die On Time sein muss. \\
Der Versuch, Zeiten im Bereich von Mikrosekunden einzustellen, scheiterte, vermutlich wegen der Trägheit der
Mechanik des Ventils. Nachfolgend wurden sukzessive "`On"'-Zeiten von 20, 30, 40ms mit "`Off"'-Zeiten von 200,
300, 400ms getestet. Die Zyklendauern wurden von 5s langsam auf bis zu 1:30min erhöht. Hier offenbarten sich
direkt Probleme mit der Suspension: obwohl sich zu Beginn immerhin die Spitze des Konus blau färbte,
wurde zwischenzeitlich scheinbar durch gröbere Partikel in der Lösung die Öffnung des Ventils verstopft,
sodass auch bei längerer Zyklusdauer der Pegel im Ventil konstant blieb. Um Ablagerungen auf dem Ventil zu
entfernen, wurde es mit DMSO ausgespült und danach für den nächsten Zyklus mit reinem DMSO befüllt. Nach dem
zweiten Durchgang öffnete sich das Ventil wieder, der Pegel sank. Vermutlich konnte sich die Öffnung des
Ventils durch die Vibrationen während des Betriebs "`freischütteln"'. Bei den nachfolgenden Messungen war
nach einer Zyklusdauer von etwa 20s das Ventil leer, sodass nur noch Luft in die Kammer gesogen wurde. Längere
Zyklen konnten also nur durchgeführt werden, indem manuell während der Messung das Ventil nachgefüllt wurde.
\\
Der Druck in der Hauptkammer war zu Beginn jedes Zyklus im Bereich von $10^{-7}$ bis $10^{-6}$. War die
Öffnung des Ventils frei und konnte die Lösung in die Kammer eintreten, fiel der Druck in der Hauptkammer in
den Bereich $10^{-4}$. Eine Druckmessung im Würfel war zunächst nicht möglich.\\
Nach diversen Versuchen mit den genannten Parametern waren auf dem Glasträger kaum blaue Partikel zu
beobachten, im Würfel hatte sich jedoch auf dem Konus, dem Boden sowie den Seitenwänden eine blaue Schicht
gebildet. Zu diesem Zeitpunkt gab die Membranpumpe am Würfel eine Fehlermeldung, vermutlich war das
zu pumpende Gasvolumen, verursacht durch das DMSO, zu hoch. Gleichzeitig war zu vermuten, dass auch die
Öffnung des Konus verstopft sein könnte, also wurde der Würfel zunächst auseinandergebaut und gereinigt.\\


