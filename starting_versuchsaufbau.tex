\subsection{Aufbau der UHV-Apparatur}

Der Sprühbeschichter wurde auf der bisher vorhandenen Ultrahochvakuums-Apparatur (siehe Abb \ref{uhvaufbau})
aufgebaut.
Diese besteht aus mehrerern Kammern, die untereinander verbunden sind. Die gesamte Apparatur ist auf vier
Druckluftfüßen gelagert, um über den Boden übetragende störende Schwingungen zu minimieren. Um ein
Ultrahochvakuum im Bereich von $10^(-10)$ zu erzeugen, kann die Apparatur zunächst mit einem mobilen Pumpstand 
(blablabla pumpe) auf Hochvakuum im Bereich von $10^{-xyz}$ gepumpt werden; ab einem Druck von $10^-6$ lässt
sich die Ionengetterpumpe hinzuschalten.\\
Die Apparatur besteht im wesentlichen aus drei wichtigen Kammern (siehe Abb. \ref{img:uhvkammern}): der 
Molekülkammer (1), der STM-Kammer (2) und der Hauptkammer (3). In der Molekülkammer kann die eingebrachte
Probe mit Hilfe des Molekülverdampfers (der in dieser Arbeit nicht zum Einsatz kam) bedampft werden. Im Rahmen
dieser Arbeit wurde der Sprühbeschichter auf die Molekülkammer aufgesetzt. In der STM-Kammer kann die Probe
mittels Rastertunnelmikroskopie- und spektroskopie untersucht sowie geflasht werden. Die Haupkammer dient zur
Untersuchung der Probe durch LEED, weiterhin kann hier durch einen unten in der Kammer eingebauten
Metallverdampfer selbige mit verschiedenen Metallen bedampft werden (Gold, Eisen, Kobalt). Auch hier ist ein
Flashen der Probe möglich über ein Heizfilament am sogenannten Manipulator, in dem die Probe gehalten wird.
Der Manipulator ermöglicht ein Drehen sowie iein Verschieben der Probe in x-z-Richtung.\\



\subsection{Aufbau des Sprühbeschichters}

Der Sprühbeschichter funktioniert im wesentlichen über ein Druckgefälle innerhalb seines Aufbaus, durch das
die eingebrachten Moleküle auf das Substrat "`gesaugt"' werden. 

Des ist ein Versuch für git. nummer 2.
und jetzt nummer 3.