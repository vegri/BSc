\subsection{Aufbau der UHV-Apparatur}

Der Sprühbeschichter soll auf der bisher vorhandenen Ultrahochvakuums-Apparatur (siehe Abb
\ref{uhvaufbau}) aufgebaut werden.%, in der das Substrat präpariert werden kann.
Diese besteht aus mehrerern Kammern, die untereinander verbunden sind. Die gesamte Apparatur ist auf vier
Druckluftfüßen gelagert, um über den Boden übetragende störende Schwingungen zu minimieren. Um ein
Ultrahochvakuum im Bereich von $10^(-10)$mbar zu erzeugen, kann die Apparatur zunächst mit einem
mobilen Pumpstand (hier verwendet: HiCube von Pfeiffer ) auf Hochvakuum im Bereich von $10^{-8}$mbar
gepumpt werden; ab einem Druck von $10^-6$mbar lässt sich eine fest installierte Ionengetterpumpe
hinzuschalten.\\
Die Apparatur besteht im wesentlichen aus drei wichtigen Kammern (siehe Abb. \ref{img:uhvkammern}): der 
Molekülkammer (1), der STM-Kammer (2) und der Hauptkammer (3). In der Molekülkammer kann die eingebrachte
Probe mit Hilfe des Molekülverdampfers (der in dieser Arbeit nicht zum Einsatz kam) bedampft werden.
Zukünftig soll auf diesen Teil der Kammer der im Rahmen dieser Bachelorarbeit gebaute
Sprühbeschichter aufgesetzt werden. Dieser Teil der Kammer kann über zwei Ventile von den anderen
Kammern abgetrennt werden.\\
%Im Rahmen dieser Arbeit wurde der Sprühbeschichter auf die Molekülkammer aufgesetzt. 
In der STM-Kammer kann die Probe
mittels Rastertunnelmikroskopie- und spektroskopie untersucht werden. In der Haupkammer kann die
Probe durch einen unten in der Kammer eingebauten Metallverdampfer mit
verschiedenen Metallen (Gold, Eisen, Kobalt) bedampft, durch Erhitzen von Ablagerungen auf der
Oberfläche gereinigt ("`geflasht"') sowie die Probenoberfläche durch LEED untersucht werden. Die
hier verwendete LEED-Apparatur ist ein "`Spectaleed" von Omicron. 
Weiterhin sind hier Augerspektroskopie und Kerr-Messung möglich, die jedoch hier nicht zum Einsatz
kamen.\\
Das verwendete Substrat, in diesem Fall der (XXX)Rheniumkristall, kann im Probenhalter mittels des
Transferstabs durch die Kammern transportiert werden. In die Molekülkammer kann er mit einem
weiteren, kleineren Transferstab gesetzt werden. Ein Transfer zum Probentisch des STMs in der
STM-Kammer ist mit Hilfe des sogenannten Wobblesticks möglich, mit dessen Greifzange die Probe an der Probennase gehalten und in
den STM-Tisch versetzt werden kann. In der Hauptkammer schlussendlich wird der Probenhalter in den
sogenannten Manipulator gesetzt, mit dem die Probe in alle Raumrichtungen verschoben sowie gedreht
werden kann. \\
Am Manipulator ist ein Heizfilament angebracht, über den der Kristall geflasht wird:
Dabei wird das Filament zum Glühen gebracht, die durch Glühemission emittierten Elektronen werden
durch eine an die Probe angelegte Hochspannung zum Kristall hin beschleunigt und heizen diesen dann
durch Stöße bis zum Glühen auf. In diesem Prozess werden aufgedampfte Schichten, aber auch
Ablagerungen, die sich selbst im Ultrahochvakuum (früher oder später) auf Oberflächen absetzen,
entfernt.\\
Weiterhin kann das Filament dazu benutzt werden, ohne angelegte Hochspannung die Probe einfach nur
zu erwärmen ("`tempern"'). Dies kann die Oberflächenstrukturen nach Aufdampfen eines Metalls
deutlich ändern. Die erreichte Temperatur lässt sich aus dem Filamentstrom bestimmen.\\
 Über dem Metallverdampfer in der Hauptkammer befindet sich eine
Schutzklappe, sodass die darüber befindliche Probe durch Auf- oder Zuklappen je nach Bedarf bei laufendem Verdampfer bedampft werden
kann. Die Menge des verdampften Metalls wird über einen Schwingquarz gemessen, der sich über dem
Verdampfer befindet und somit mitbedampft wird; anhand der Änderung der Schwingfrequenz können
Rückschlüsse auf die abgelagerte Masse auf dem Quarz gezogen werden, worüber dann die Anzahl der
aufgedampften Lagen auf der Probe bestimmt werden kann. Nach Skalierung wird hierfür die digital
angezeigte Periodendauer genutzt, dabei entsprechen für Gold z.B. 250 Skaltenteile etwa einer
Monolage Gold auf der Probe.\\
Restgasanalyse, druckmessung


%  Auch hier ist ein
% Flashen der Probe möglich über ein Heizfilament am sogenannten Manipulator, in dem die Probe gehalten wird.
% Der Manipulator ermöglicht ein Drehen sowie ein Verschieben der Probe in x-z-Richtung.\\



\subsection{Aufbau des Sprühbeschichters}

Der Sprühbeschichter funktioniert im wesentlichen über ein Druckgefälle innerhalb seines Aufbaus, durch das
die in einer Lösung gelösten Moleküle auf das Substrat "`gesaugt"' werden und sie dort im besten
Falle haften bleiben. 
% \begin{figure}[H]
% \centering
% \includesvg[width=0.5\textwidth]{aufbau}
% \caption{ }
% \label{aufbau}
% \end{figure}
Ein Schema des Aufbaus ist in Abb. \ref{aufbau} zu sehen. Ganz oben befindet sich ein elektrisch
gesteuertes Ventil, in dessen Reservoir sich die Suspension befindet. Dieses Ventil wird über eine
Steuerungseinheit mit verschiedenen einstellbaren Öffnungsmodi bedient (siehe Abschnitt \ref{kaptest}). Ist es
geöffnet, wird die Suspension durch die 1mm große Öffnung in den darunterliegenden, vakuumierten Würfel
gesogen. Das Vakuum in diesem Würfel wird durch einen an einer Seite angeschlossenen Pumpstand hergestellt
und soll idealerweise im Bereich von $10^{-5}$mbar liegen. Im Würfel befindet sich ein metallischer
Konus, dessen Spitze eine Öffnung von wiederum 0,1mm aufweist. Er ist die Verbindung zwischen Würfel und dem
unteren Teil des Aufbaus, der auf der Molekülkammer aufgesetzt ist und einen Druck im Bereich von
$10^{-8}$mbar aufweisen sollte.  
Seine Aufgabe ist, dafür zu sorgen, dass einerseits nur ein dünner Strahl der eingelassenen Suspension in
den unteren Teil des Aufbaus gelangt, damit sich die Moleküle gleichmäßig auf dem Substrat verteilen und
nicht in Tröpfchen. Gleichzeitig soll er durch seine kleine Öffnung einen schnellen Druckausgleich zwischen
dem unteren Teil und dem Würfel verhindern. \\
Weiterhin sind zwei weitere Ventile in dem Aufbau enthalten. Mit Ventil 1 kann der Würfel von der
Molekülkammer abgetrennt werden. Dies kann man nutzen, um Würfel oder das elektrische Ventil zu säubern, ohne
dabei der kompletten vorderen Teil der Kammer belüften zu müssen. Wird der Würfel dann bei geschlossenem
Ventil 1 erneut über den Pumpstand vakuumiert, kann Ventil 2 geöffnet werden, damit auch das Verbindungsstück
unterhalb des Würfels über Schlauch und T-Stück mitgepumpt werden kann und nicht nur über die kleine Öffnung
des Konus.\\
Das zweite Verbindungsstück ist eine Überbrückung zwischen Ventil 1 und Adapterflansch, mit dem
der 40mm-Anschluss von Ventil 1 %bzw. des zweiten Verbindungsstückes 
auf den 63mm-Flansch der Molekülkammer gesetzt werden kann. Eine direkt Verbindung zwischen diesem
Ventil und dem Adapter ist nicht möglich, da beide ein geschlossenes Gewinde (?) besitzen.\\
Unten in der Molekülkammer befindet sich das Substrat. Die durch den Konus in den unteren Teil des Aufbaus
gelangten Moleküle sollten sich in der restlichen Kammer verteilen und auf dem Substrat zum Liegen kommen.

% An einer Seite des Würfels ist ein Pumpstand angeschlossen, der ein Vakuum im
% Bereich von $10^{-5}$mbar herstellen soll. Der Teil unterhalb des Würfels ist mit der Molekülkammer
% verbunden und wird über die Getterpumpe vakuumiert. Lediglich zu Beginn, wenn der Pumpstand
% hochgefahren wird, kann das zweite Ventil geschlossen und das erste geöffnet werden, sodass das
% erste Verbindungsstück direkt unterhalb des Würfels über Schlauch und T-Stück ebenfalls gepumpt werden
% kann.\\
% Das erste Ventil dient auch dazu, den Würfel von der Kammer trennen zu können, falls dieser
% oder das Ventil für die Suspension gereinigt werden müssen. Auf diese Weise muss nicht der ganze
% vordere Teil der Vakuumkammer, sondern nur der Würfel belüftet werden.\\



