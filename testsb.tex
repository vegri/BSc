Der Aufbau zur Spraydeposition wurde an einer kleineren HV-Apparatur getestet (siehe \ref{testapparatur}). Der
Vorteil ist, dass diese Kammer wegen ihres kleineren Volumens schneller abgepumpt werden kann, was bei
auftretenden Problemen, bei denen man die Kammer öffnen muss, ein schnelleres Arbeiten ermöglicht.\\
Die sogenannte Testapparatur besteht aus zwei Pumpen (Vorpumpe (1), Marke bla, und Hauptpumpe (2), Marke bla),
einer Hauptkammer mit Massenspektrometer (3) und Druckmessgerät bla (4) sowie dem aufgesetzten
Sprühbeschichter (4). Am Würfel des Beschichters ist eine (weitere) Membranpumpe angebracht (5). Das Ventil
zum Einspritzen des aufzutragenden Moleküls (6) wird angesteuert über ein Netzteil von IotaOne (7). Das
Substrat, hier zum Test ein Glasträger, befindet sich unterhalb des Beschichters (7.5) und kann durch einen
Fensterflansch beobachtet werden.\\
Die Ansteuerung des Ventils ermöglicht
prinzipiell zwei Modi: ein einmaliges Öffnen des
Ventils für einen einstellbaren Zeitraum (One
Shot Modus) und einen Zyklus von abwechselndem
Öffnen und Schließen des Ventils (Cycle Modus).
Da ein Einspritzen der Lösung in das Vakuum
natürlich den Druck steigen lässt, wurde hier
zunächst der Cycle Modus getestet, da so
mutmaßlich der Druck durch das Schließen des
Ventils zwischendrin besser erhalten werden
kann. In diesem Modus müssen die Dauern für das
Öffnen (On-Time) und des Schließens (Off-Time)
gewählt werden, wobei die Off-Time
betriebsbedingt mindestens so lang wie die
On-Time sein muss.\\
Beim Vakuumieren der Apparatur wurde Ventil 1
 geschlossen und Beschichter sowie die 
 Hauptkammer getrennt gepumpt, bis beide
 Vakuumpumpen jeweils vollständig hochgefahren
 waren.
 Um ein Druckgefälle zwischen Würfel und unterem
 Teil des Aufbaus herzustellen, getrennt durch den
 Konus, wurde dann das Ventil 2 geschlossen und
 das Ventil 1 geöffnet.
 Somit sollte innerhalb des Konus und dem
 Verbindungsstück darunter annähernd der gleiche
 Druck herrschen wie im Rest der Haupkammer.
 Nun konnte das gelöste Molekül über eine Pipette
 in das Reservoir des elektrischen Ventils
 eingefüllt und die On- sowie Off-Time gewählt
 werden.\\
