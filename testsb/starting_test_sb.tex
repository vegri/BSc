Der Aufbau zur Spraydeposition wurde an einer kleineren HV-Apparatur getestet. %\\(siehe
%\ref{testapparatur}).
Der
Vorteil ist, dass diese Kammer wegen ihres kleineren Volumens schneller abgepumpt werden kann, was bei
auftretenden Problemen, bei denen man die Kammer öffnen muss, ein schnelleres Arbeiten ermöglicht.\\
Die sogenannte Testapparatur besteht aus einer Vorpumpe und  einer Hauptpumpe,
der Hauptkammer mit Massenspektrometer und Druckmessgerät sowie dem aufgesetzten
Sprühbeschichter. Am Würfel des Beschichters ist ein mobiler Pumpstand angeschlossen. Das Ventil
zum Einlassen des aufzutragenden Moleküls wird angesteuert über ein Netzteil von IotaOne. Das
Substrat, hier zum Test ein Glasträger, befindet sich unterhalb des Beschichters und kann durch einen
Fensterflansch beobachtet werden. Die hier eingesetzten Moleküle waren Kupfer-2-phthalocyanin (kurz
CuPc), ein blauer Feststoff, der als Pigment
beispielsweise in Kunststoffen, Lacken oder
Druckerfarben verwendet wird, sowie 
Tetracyanoquinodimethan (kurz TCNQ), eine gelbfarbene,
organische Verbindung, die u.a. als Akzeptor zur Präparation von Ladungstransfersalzen genutzt wird.
Beide Stoffe sollten aufgrund ihrer Farbe auf dem Glasträger gut zu erkennen sein.

 
\begin{figure}[H]
	\centering
	\sffamily
	\includesvg[svgpath=testsb/]{sb}
	\caption{\textit{Der Aufbau des Sprühbeschichters auf der Testapparatur. Statt wie in Abb.
	\ref{aufbau} mussten hier Ventil 1 und 2 sowie das T-Stück um jeweils $90^{\circ}$ versetzt
	angebracht werden. Als Testsubstrat wird hier ein Glasträger verwendet, der durch einen
	Fensterflansch beobachtet werden kann. }}
\label{aufbau}
\end{figure}
 
 

Die Ansteuerung des elektrischen Ventils ermöglicht
prinzipiell zwei Modi: ein einmaliges Öffnen des
Ventils für einen einstellbaren Zeitraum (One
Shot Modus) und einen Zyklus von abwechselndem
Öffnen und Schließen des Ventils (Cycle Modus).
Da ein Einlassen der Lösung in das Vakuum
natürlich vorübergehend den Druck steigen lässt, wurde hier
zunächst der Cycle Modus getestet, da so
mutmaßlich der Druck durch das Schließen des
Ventils zwischendrin besser erhalten werden
kann. In diesem Modus müssen die Dauern für das
Öffnen (On-Time) und des Schließens (Off-Time)
gewählt werden, wobei die Off-Time
betriebsbedingt mindestens so lang wie die
On-Time sein muss.\\
Beim Vakuumieren der Apparatur wurde Ventil 1
 geschlossen und Beschichter sowie die 
 Hauptkammer getrennt gepumpt, bis beide
 Vakuumpumpen jeweils vollständig hochgefahren
 waren.
 Um ein Druckgefälle zwischen Würfel und unterem
 Teil des Aufbaus herzustellen, getrennt durch den
 Konus, wurde dann Ventil 2 geschlossen und
 Ventil 1 geöffnet.
 Somit sollte innerhalb des Konus und dem
 Verbindungsstück darunter annähernd der gleiche
 Druck herrschen wie im Rest der Haupkammer.
 Nun konnte das gelöste Molekül über eine Pipette
 in das Reservoir des elektrischen Ventils
 eingefüllt und die On- sowie Off-Time gewählt
 werden.\\
 Der Versuch, On- und Off-Zeiten für das Ventil im
Bereich von wenigen Mikrosekunden einzustellen,
scheiterte, vermutlich wegen der Trägheit der Mechanik des
Ventils. Es wurden dann verschiedene On-Zeiten im Bereich von 1ms bis zu 50ms getestet, bei
Off-Zeiten im Bereich von 7ms bis 4s. Trotz vieler Versuche und langen Versuchszyklen bis zu 15min
war bei diesen Tests nichts auf dem Glasträger zu sehen, obwohl ein Teil der Suspension durch den
Konus in die Hauptkammer gelangt sein musste - zu sehen am Druckanstieg von etwa $10^{3}$mbar in
selbiger. Es ergaben sich jedoch eine Reihe von anderen bemerkenswerten Beobachtungen.\\
Zunächst sind sowohl Ventil als auch die Öffnung des Konus sehr anfällig für Verstopfungen. Dass im
Würfel eine Blockade vorliegt merkt man daran, dass im Hauptteil der Kammer der
Druck beim Öffnen des elektrischen Ventils nicht mehr ansteigt, sondern stagniert bzw. weiterhin
langsam fällt. Dieses Problem ergab sich vor allem mit CuPc, das gelöst in Dimethylsulfoxid eine
grobe Suspension mit sichtbar großen, ungelösten Partikeln ergab. Ein Wechsel des Lösungsmittels auf
Dimethylformamid brachte keine spürbare Verbesserung. Zur groben Säuberung des Ventils wurden
zunächst ein paar Zyklen mit reinem Lösungsmittel im Reservoir durchgeführt, was jedoch, wie bei
einem Abbau und Auseinandernehmen des Ventils sichtbar wurde, die Ablagerungen im Inneren des
Ventils nicht entfernen kann. Zusätzlich werden die Pumpen unnötig mit dem abzupumpenden Gas
belastet. Um das Ventil daher wirkungsvoll zu säubern, muss es abgeschraubt und auseinander gebaut
werden. Auch ein Säubern des Konus erfordert ein Unterbrechen des Vakuums. Bevor der Konus jedoch
dazu komplett aus dem Würfel entfernt wird, wobei der Würfel inklusive Ventil 1 und T-Stück
abgeschraubt werden muss, kann man versuchen, einen Flansch an einer Seite des Würfels abzuschrauben
und den Konus an der Spitze mit Lösungsmittel abzuwischen. Dies ist weniger aufwendig und
erfüllt meistens seinen Zweck.\\
Der am Würfel angebrachte Pumpstand war mit den zu pumpenden Gasvolumina relativ schnell
überfordert. Zu sehen war dies am Stromverbrauch, die je nach eingestellter On-Time nach
wenigen Minuten auf das zwei- bis dreifache des üblichen Wertes anstieg; wenige Male schaltete sie
sich mit einer Fehlermeldung selbst aus. Das Problem konnte gelöst werden, indem die interne
Membranvorpumpe des Pumpstand gegen eine externe Pumpe, die für höhere Gasvolumina ausgelegt ist,
ausgetauscht wurde.
Auf lange Sicht sollte eventuell eine Pumpe gefunden werden, die vom Leistungsvermögen für diesen
Aufbau geeignet ist.\\
Ein weiterer wichtiger Punkt bei diesem Aufbau ist die Druckerhaltung in der Kammer sowie im Würfel.
Der Sprühbeschichter ist dazu gedacht, auf den vorderen Teil der UHV-Apparatur aufgesetzt zu werden,
in der der Druck im Bereich von $10^{-8}$mbar liegt. Getestet wurde der Beschichter im Vakuum von
$10^{-5}$mbar. Sollte hier der Druck signifikant ansteigen, wird er in der UHV-Apparatur um viele
Größenordnungen nach oben springen. Je nachdem wie lange es dauert, den vorderen Teil wieder in den
Bereich von $10^{-8}$mbar zu vakuumieren, damit die Verbindungen zur anderen Kammer geöffnet werden
können, verliert die Probenpräparation möglicherweise an Qualität. Von daher ist es von großem
Interesse, den Druckanstieg beim Öffnen des Ventils möglichst gering zu halten.\\
Dies ist nur zum Teil gelungen. Nachdem ein weiteres Druckmessgerät am Würfel angebracht wurde, war
schnell zu sehen, dass der Druck hier gerade mal etwa $10^{-2}$mbar betrug, aber auch nur in diesem
Bereich schwankte. In der Hauptkammer stieg der Druck zu Beginn um bis zu drei Größenordnungen, wenn
die On-Time zu groß gewählt wurde. Über die Off-Time konnte zumindest geregelt werden, dass sich der
Druck innerhalb eines On-Off-Zyklusses vor dem erneuten Öffnen des Ventils wieder auf den
Anfangswert einpendelte. Offensichtlich scheinen eine kurze On-Time und eine im Vergleich lange
Off-Time der richtige Weg zu sein.\\
Da zur Vermutung stand, dass der Weg zwischen Konusspitze und Glasträger für die wenigen durch die
Konusöffnung gelangten Moleküle zu lang war, wurde der Aufbau nach und nach ein wenig verändert:
Zuerst wurde testweise Ventil 1 und das zweite Verbindungsstück entfernt, das erste Verbindungsstück
unterhalb des Würfels also direkt auf den Adapterflansch gesetzt. Beim nächsten Umbau wurde die
Pumpe am Würfel entfernt, ebenso der Konus; schlussendlich kamen erste sichtbare Ergebnisse erst,
als der komplette Aufbau entfernt, das elektrische Ventil einfach auf den
Adapterflansch gesetzt und eine On-Time von 1,5ms eingestellt wurde. Hierbei ergaben sich
ungleichmäßig verteilte farbige Spritzer auf dem Glasträger.\\
Das Ergebnis konnte reproduziert werden mit dem Aufbau aus Würfel und Konus, jedoch ohne den am
Würfel eigentlich angeschlossenen Pumpstand. Bei einer On-Time von 1,5ms, einer Off-Time von
400ms und einem Anfangsdruck von ca $2\cdot10^{-5}$mbar blieb der Druck auch im Bereich von
$10^{-5}$mbar.\\
Unglücklicherweise konnte nichts mehr auf dem Glasträger gesehen werden, wenn der ursprüngliche
Zustand des Sprühbeschichters wiederhergestellt und der Pumpstand wieder angeschlossen wurde. Da
dennoch aufgrund des Druckanstiegs, auch wenn er mittlerweile durch passende On/Off-Zeiten minimal
war, etwas in die Hauptkammer gelangen musste, stand nun zur Vermutung, dass die Schichten, die
sich dabei auf dem Glasträger absetzen sollten, einfach zu dünn sind, um sie mit bloßem Auge
erkennen zu können.\\
Abhilfe sollte nun eine Quarzwaage schaffen, mit der sich anhand der Frequenzänderung auch Monolagen
von Atomen bzw. Molekülen nachweisen lassen (siehe Kap. \ref{kapaufbau}). Die Quarzwaage wurde
direkt unter dem Adapterflansch, also über dem Glasträger angebracht und an einen Frequenzgeber und einer
digitalen Anzeige der gemessenen Periodendauer angeschlossen.\\
Leider konnte in der verbliebenen Laborzeit nur eine Messung durchgeführt werden. Bei einer
On-Time von 3ms und einer Off-Time von 500ms stieg die Periodendauer über die Dauer von etwa einer
halben Stunde gleichmäßig um $30\pm1$ Skalenteile an, sogar bei gleichbleibendem Druck in der
Hauptkammer. Danach stagnierte der Anstieg, offensichtlich wegen einer Blockade im Würfel (der
Druck fiel zu diesem Zeitpunkt weiter ab). Ventil 1 wurde geschlossen, das Ventil abgebaut und
gereinigt. In dieser Zeit sank die Periodendauer wieder um ca 15 Skaltenteile und stagnierte dann,
was vermutlich auf das Verdampfen des mitaufgebrachten Lösungsmittels zurückzuführen ist.\\
Zwar blieb es zunächst bei diesem einen Versuch, dennoch kann man daraus schließen, dass der
Versuchsaufbau prinzipiell funktioniert. Da auf dem eigentlichen Substrat, dem Re-Kristall,
letztendlich nur Monolagen aufgebracht und untersucht werden sollen, ist der
Versuchsaufbau in seiner jetzigen Form eventuell sogar schon ausreichend. Dies sollte vorab
allerdings durch weitere Versuche mit der Quarzwaage an der Testapparatur tiefergehend untersucht
werden.


























