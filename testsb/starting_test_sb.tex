Der Aufbau zur Spraydeposition wurde an einer kleineren HV-Apparatur getestet. %\\(siehe
%\ref{testapparatur}).
Der
Vorteil ist, dass diese Kammer wegen ihres kleineren Volumens schneller abgepumpt werden kann, was bei
auftretenden Problemen, bei denen man die Kammer öffnen muss, ein schnelleres Arbeiten ermöglicht.\\
Die sogenannte Testapparatur besteht aus einer Vorpumpe und  einer Hauptpumpe,
der Hauptkammer mit Massenspektrometer und Druckmessgerät sowie dem aufgesetzten
Sprühbeschichter. Am Würfel des Beschichters ist ein mobiler Pumpstand angeschlossen. Das Ventil
zum Einlassen des aufzutragenden Moleküls wird angesteuert über ein Netzteil von IotaOne. Das
Substrat, hier zum Test ein Glasträger, befindet sich unterhalb des Beschichters und kann durch einen
Fensterflansch beobachtet werden. Die hier eingesetzten Moleküle waren Kupfer-2-phthalocyanin (kurz
CuPc), ein blauer Feststoff, der als Pigment
beispielsweise in Kunststoffen, Lacken oder
Druckerfarben verwendet wird, sowie 
Tetracyanoquinodimethan (kurz TCNQ), eine gelbfarbene,
organische Verbindung, die u.a. als Akzeptor zur Präparation von Ladungstransfersalzen genutzt wird.
Beide Stoffe sollten aufgrund ihrer Farbe auf dem Glasträger gut zu erkennen sein.

 
\begin{figure}[H]
	\centering
	\sffamily
	\includesvg[svgpath=testsb/]{sb}
	\caption{\textit{Der Aufbau des Sprühbeschichters auf der Testapparatur. Statt wie in Abb.
	\ref{aufbau} mussten hier Ventil 1 und 2 sowie das T-Stück um jeweils $90^{\circ}$ versetzt
	angebracht werden. Als Testsubstrat wird hier ein Glasträger verwendet, der durch einen
	Fensterflansch beobachtet werden kann. }}
\label{aufbau}
\end{figure}
 
 

Die Ansteuerung des elektrischen Ventils ermöglicht
prinzipiell zwei Modi: ein einmaliges Öffnen des
Ventils für einen einstellbaren Zeitraum (One
Shot Modus) und einen Zyklus von abwechselndem
Öffnen und Schließen des Ventils (Cycle Modus).
Da ein Einlassen der Lösung in das Vakuum
natürlich vorübergehend den Druck steigen lässt, wurde hier
zunächst der Cycle Modus getestet, da so
mutmaßlich der Druck durch das Schließen des
Ventils zwischendrin besser erhalten werden
kann. In diesem Modus müssen die Dauern für das
Öffnen (On-Time) und des Schließens (Off-Time)
gewählt werden, wobei die Off-Time
betriebsbedingt mindestens so lang wie die
On-Time sein muss.\\
Beim Vakuumieren der Apparatur wurde Ventil 1
 geschlossen und Beschichter sowie die 
 Hauptkammer getrennt gepumpt, bis beide
 Vakuumpumpen jeweils vollständig hochgefahren
 waren.
 Um ein Druckgefälle zwischen Würfel und unterem
 Teil des Aufbaus herzustellen, getrennt durch den
 Konus, wurde dann Ventil 2 geschlossen und
 Ventil 1 geöffnet.
 Somit sollte innerhalb des Konus und dem
 Verbindungsstück darunter annähernd der gleiche
 Druck herrschen wie im Rest der Haupkammer.
 Nun konnte das gelöste Molekül über eine Pipette
 in das Reservoir des elektrischen Ventils
 eingefüllt und die On- sowie Off-Time gewählt
 werden.\\
 Der Versuch, On- und Off-Zeiten für das Ventil im
Bereich von wenigen Mikrosekunden einzustellen,
scheiterte, vermutlich wegen der Trägheit der Mechanik des
Ventils. Es wurden dann verschiedene On-Zeiten im Bereich von 1ms bis zu 50ms getestet, bei
Off-Zeiten im Bereich von 7ms bis 4s. Trotz vieler Versuche und langen Versuchszyklen bis zu 15min
war bei diesen Tests nichts auf dem Glasträger zu sehen, obwohl ein Teil der Suspension durch den
Konus in die Hauptkammer gelangt sein musste - zu sehen am Druckanstieg von etwa $10^{3}$mbar in
selbiger. Es ergaben sich jedoch eine Reihe von anderen bemerkenswerten Beobachtungen.\\
Zunächst sind sowohl Ventil als auch die Öffnung des Konus sehr anfällig für Verstopfungen. Dass im
Würfel eine Blockade vorliegt, merkt man daran, dass im Hauptteil der Kammer der
Druck bein Öffnen des elektrischen Ventils nicht mehr ansteigt, sondern stagniert bzw. weiterhin
langsam fällt. Dieses Problem ergab sich vor allem mit CuPc, das gelöst in Dimethylsulfoxid eine
grobe Suspension mit sichtbar großen, ungelösten Partikeln ergab. Ein Wechsel des Lösungsmittels auf
Dimethylformamid brachte keine spürbare Verbesserung. Zur groben Säuberung des Ventils wurden
zunächst ein paar Zyklen mit reinem Lösungsmittel im Reservoir durchgeführt, was jedoch, wie bei
einem Abbau und Auseinandernehmen des Ventils sichtbar wurde, die Ablagerungen im Inneren des
Ventils nicht entfernen kann. Zusätzlich werden die Pumpen unnötig mit dem abzupumpenden Gas
belastet. Um das Ventil daher wirkungsvoll zu säubern, muss es abgeschraubt und auseinander gebaut
werden. Auch ein Säubern des Konus erfordert ein Unterbrechen des Vakuums. Bevor der Konus jedoch
dazu komplett aus dem Würfel entfernt wird, wobei der Würfel inklusive Ventil 1 und T-Stück
abgeschraubt werden muss, kann man versuchen, einen Flansch an einer Seite des Würfels abzuschrauben
und den Konus einfach an der Spitze mit Lösungsmittel abzuwischen. Dies ist weniger aufwendig und
erfüllt meistens seinen Zweck.\\


\\
\\
 500$\mu$s
 






























