Der Aufbau zur Spraydeposition wurde an einer kleineren HV-Apparatur getestet (siehe \ref{testapparatur}). Der
Vorteil ist, dass diese Kammer wegen ihres kleineren Volumens schneller abgepumpt werden kann, was bei
auftretenden Problemen, bei denen man die Kammer öffnen muss, ein schnelleres Arbeiten ermöglicht.\\
Die sogenannte Testapparatur besteht aus zwei Pumpen (Vorpumpe (1), Marke bla, und Hauptpumpe (2), Marke bla),
einer Hauptkammer mit Massenspektrometer (3) und Druckmessgerät bla (4) sowie dem aufgesetzten
Sprühbeschichter (4). Am Würfel des Beschichters ist eine (weitere) Membranpumpe angebracht (5). Das Ventil
zum Einspritzen des aufzutragenden Moleküls (6) wird angesteuert über ein Netzteil von IotaOne (7). Das
Substrat, hier zum Test ein Glasträger, befindet sich unterhalb des Beschichters (7.5) und kann durch einen
Fensterflansch beobachtet werden.\\
 Zu Beginn der Testreihe wurde das große Ventil (8) geschlossen und Beschichter sowie die Hauptkammer getrennt
 gepumpt, bis beide Vakuumpumpen vollständig hochgefahren waren. Um ein Druckgefälle zwischen Würfel und
 Hauptkammer herzustellen, wurde dann das kleine Ventil (9) geschlossen und das große Ventil geöffnet. Nun
 konnte das gelöste Molekül über eine Pipette in das obere Ventil (?) eingefüllt und dieses über das Netzteil
 über einen einstellbaren Zeitraum geöffnet werden.\\
Bei dem zum Testen benutzten Molekül handelte es sich um Kupfer-2-phthalocyanin (kurz CuPc), ein blauer
Feststoff, der als Pigment beispielsweise in Kunststoffen, Lacken oder Druckerfarben verwendet wird. Das CuPc
wurde in dem farblosen organischen Lösungsmittel Dimethylsulfoxid (kurz DMSO) aufgelöst, woraus sich eine
grobe Suspension ergab. Nachdem diese in das oberste Ventil hinein geträufelt wurde, mussten geeignete
Parameter zur Öffnung des Ventils gefunden werden. Die Ansteuerung ermöglicht prinzipiell zwei Modi: ein
einmaliges Öffnen des Ventils für einen einstellbaren Zeitraum (One Shot Modus) und einen Zyklus von
abwechselndem Öffnen und Schließen des Ventils (Cycle Modus). Da ein Einspritzen der Lösung in das Vakuum
natürlich den Druck steigen lässt, wurde hier (zunächst) der Cycle Modus gewählt, da so mutmaßlich der Druck
durch das Schließen des Ventils zwischendurch besser erhalten werden kann. In diesem Modus müssen die Zeiten
für das Öffnen (On-Time) und des Schließens (Off-Time) gewählt werden, wobei die Off-Time
betriebsbedingt (?) mindestens so lang wie die On-Time sein muss. \\
Der Versuch, Zeiten im Bereich von Mikrosekunden einzustellen, scheiterte, vermutlich wegen der Trägheit der
Mechanik des Ventils. Nachfolgend wurden sukzessive "`On"'-Zeiten von 20, 30, 40ms mit jeweils gleich langen
"`Off"'-Zeiten getestet. Die Zyklendauern wurden von 5s langsam auf bis zu 1:30min erhöht. Hier offenbarten
sich direkt Probleme mit der Suspension: obwohl sich zu Beginn immerhin die Spitze des Konus blau färbte,
wurde zwischenzeitlich scheinbar durch gröbere Partikel in der Lösung die Öffnung des Ventils verstopft,
sodass auch bei längerer Zyklusdauer der Pegel im Ventil konstant blieb. Um Ablagerungen auf dem Ventil zu
entfernen, wurde es mit DMSO ausgespült und danach für den nächsten Zyklus mit reinem DMSO befüllt. Nach dem
zweiten Durchgang öffnete sich das Ventil wieder, der Pegel sank. Vermutlich konnte sich die Öffnung des
Ventils durch die Vibrationen während des Betriebs "`freischütteln"'. Bei den nachfolgenden Messungen war
nach einer Zyklusdauer von etwa 20s das Ventil leer, sodass nur noch Luft in die Kammer gesogen wurde. Längere
Zyklen konnten also nur durchgeführt werden, indem manuell während der Messung das Ventil nachgefüllt wurde.
\\
Der Druck in der Hauptkammer war zu Beginn jedes Zyklus im Bereich von $10^{-7}$ bis $10^{-6}$. War die
Öffnung des Ventils frei und konnte die Lösung in die Kammer eintreten, fiel der Druck in der Hauptkammer in
den Bereich $10^{-4}$. Eine Druckmessung im Würfel war zunächst nicht möglich.\\
Nach diversen Versuchen mit den genannten Parametern waren auf dem Glasträger kaum blaue Partikel zu
beobachten, im Würfel hatte sich jedoch auf dem Konus, dem Boden sowie den Seitenwänden eine blaue Schicht
gebildet. Zu diesem Zeitpunkt gab die Membranpumpe am Würfel eine Fehlermeldung, vermutlich war das
zu pumpende Gasvolumen, verursacht durch das DMSO, zu hoch. Gleichzeitig war zu vermuten, dass auch die
Öffnung des Konus verstopft sein könnte, also wurde der Würfel zunächst auseinandergebaut und gereinigt.\\
Beim Reinigen fiel auf, dass der Konus innen etwa auf dem ersten Centimeter eine blaue Schicht aufwies, der
Rest des Konus und der restliche Weg bis zum Glasträger wies keine augenscheinlichen Blaufärbungen auf. Die
nächste Idee war es, die CuPc-DMSO Lösung zu filtern, um grobe Partikel nicht mit in das Vakuum einzubringen.
Die vom Institut für Organische Chemie der Uni Mainz erhaltenen Filterpapiere waren jedoch  trotz gröbster
Stufe immer noch zu fein, sodass nach dem Filtern nur eine schwach bläuliche Flüssigkeit übrig blieb. Diese
Flüssigkeit wurde nicht getestet, da die Vermutung nahe lag, dass das Verhältnis Lösung zu Farbstoff zu hoch
war - und durch die viele Flüssigkeit nur die Pumpe belastet würde, ohne dass es das wenige CuPc bis zum
Glasträger schaffen würde.\\
Stattdessen wurde weiter die Suspension mit der gröberen Partikeln verwendet, jedoch die interne Vorpumpe im
Pumpstand für den Würfel abgeschraubt und eine andere, für höhere Gasvolumina ausgelegte Vorpumpe extern
angeschlossen. Zudem wurde die On-Time auf 1ms eingestellt, die Off-Time jedoch auf 50ms, später auf 100ms, um
den Druck insgesamt besser erhalten zu können. Außerdem wurde an den Würfel ein Druckmessgerät angeschlossen.
Bei den folgenden Messungen fiel der Druck von anfänglich $10^{-5}$ in den $10^{-3}$-Bereich in der
Hauptkammer, im Würfel schwankte der Druck hauptsächlich im $10^{-2}$-Bereich. Nach fünf Durchgängen war der
Strom bei beiden Pumpen auf über 2A gestiegen, sie somit also eine sehr hohe Leistung erbringen mussten, und
weiterhin war nichts auf dem Glasträger zu sehen.\\
Als nächstes wurde das Lösungsmittel getauscht, das DMSO wurde durch Dimethylformamid (kurz DMF) ersetzt,
da dieses das CuPc besser lösen sollte. Leider blieben noch immer grobe Partikel in der Lösung übrig. Weiterhin
wurden die On/Off-Zeiten geändert und die On-Time immer weiter verkürzt, bis zu 1ms On und 3s Off. Dies
bewirkte, dass innerhalb eines Ventil-auf-Ventil-zu-Zyklusses der Druck zwar anstieg bis in den
$10^{-3}$-Bereich, sich jedoch vor dem erneuten Öffnen des Ventils wieder in den Anfangsbereich $10^{-5}$
regulierte. Im Würfel schwankte der Druck wie vorher nur im mittleren $10^{-2}$-Bereich.\\
Nachdem sich immer noch keine sichtbare Schicht auf dem Glasträger absetzte, wurde der Aufbau kurzerhand
geändert: Das große Ventil (bla) wurde abgeschraubt und direkt an den Zugang zum Pumpstand (bla) gelegt.
Kammer und Würfel können auf diese Weise zwar nicht mehr komplett getrennt gepumpt werden, jedoch sollte
weiterhin ein Druckgefälle von Würfel und Kammer möglich sein. Der Vorteil dieses Aufbaus ist, dass das bisher
notwendige Verbindungsstück zwischen großem Ventil und dem Adapterflansch wegfallen konnte, sich somit der Weg
zum Glasträger verkürzt (siehe Bild \ref{bla}. \\
 Da sich nach einem Versuch mit dem geänderten Aufbau die Öffnung des Konus zusetzte - der Druck in der
 Hautkammer änderte sich gar nicht mehr, sondern fiel immer weiter - wurde beschlossen, ein anderes Molekül als CuPc zu nehmen, da die
Parikel offensichtlich nur schlecht durch die Öffnung passen. Lösungsmittel, die es noch besser lösen könnten,
fallen unter die Kategorie stark gesundsheitsschädlich und sollten wegen mangelndem Abzug in diesem Fall nicht
verwendet werden.\\
Das nächste Lösungsmittel war Tetracyanochinodimethan (TCNQ), ein gelbfarbener, organische
Halbleiter, gelöst in Tetrahydrofuran (THF), wie die bisherigen Lösungsmittel ebenfalls ein
organisches Lösungsmittel. \\
Der erste Versuch lieferte bei einer On-Time von 1ms und einer Off-Time von 4s und einer
Gesamtdauer von 15min wiederum keine Farbablagerungen auf dem Glasträger. Der Druck ändert sich im
Würfel vom $10^{-2}$ in den $10^{-1}$mbar-Bereich, in der Kammer stieg er von $10^{-5}$ wieder auf $10^{-3}$mbar. Da der
Pumpstand, angezeigt durch hohen Strom und fallender Drehzahl, das Gasvolumen nicht verarbeiten
konnte, blieb es bei diesem Versuch mit diesen Parametern.\\
Mit der Vermutung, dass die Öffnung des Konus zu klein sein könnte, wurde dieser für die nächste
Messung komplett entfernt, ebenso die Anbindung an den mobilen Pumpstand. Nun war kein Druckgefälle
in Richtung des Glasträgers mehr vorhanden, dennoch sollte sich die Suspension ins Vakuum gebracht
verteilen und sich, so die Hoffnung, auch auf den Glasträger ablagern. Da nun das Gasvolumen durch
keine zusätzliche Pumpe mehr gepumpt werden konnte, wurde, um den Druck einigermaßen zu erhalten,
die On-Time auf 500µs erniedrigt, außerdem wurde zu zunächst der One-Shot-Modus eingestellt. Der
Druck stieg in der Kammer nur innerhalb des $10^{-5}$mbar-Bereiches an. Nach fünf Versuchen war auch
hier keine Beobachtung auf dem Glasträger zu machen. \\
Da auch der Würfel bei diesem Aufbau eigentlich nicht gebraucht wird, genauso wie die Ventile und
Verbindungen, konnte die Strecke zwischen Ventil und Glasträger weiter reduziert werden, indem der
Würfel komplett ausgebaut und das Ventil direkt auf den Adapterflansch über dem Glasträger befestigt
wurde (siehe Bild /ref{blabla}). Hier wurde zuerst getestet, bei welchen Öffnungszeiten des Ventils
der Druck innerhalb der Kammer vom Startwert (i. d. R $3\cdot10^{-5}$mbar) über den Druckanstieg
zurück zum Startwert ging. Getestet wurden Zeiten in der Reihenfolge von 500, 300, 200, 250µs. Bei
200µs änderte sich der Druck in der Kammer gar nicht, sondern fiel weiter. Vermutlich lag auch 
diese On-Time außerhalb der mechanischen Reaktionsfähigkeit des Ventils. Die Zeiten bis zur
Erholung des Druckes lagen bei etwa 60s für 500µs, 50s für 300µs und stark schwankend zwischen 45s
und 70s für 250µs. Der Druck steig bei allen Versuchen auch nur innerhalb des
$10^{-5}$mbar-Bereiches an, je kürzer die On-Time, desto kleiner der Druckanstieg. Bei diesen
Versuchen war noch keine Beobachtung auf dem Glasträger zu machen. Um auszuschließen, dass dies an
der Kürze des Zeitintervalls lag, wurde eine lange Messung im Cycle-Modus über 1h und 15min
durchgeführt bei einer On-Time von 250µs und einer Off-Time von 70s, um den Druck über diesen
Zeitraum nicht konstant langsam zu steigern. Auch nach diesem Versuch war weder ein Gelbschimmer
auf dem Glasträger noch auf den Metallträgern zu sehen (diese sollten vielleicht noch erwähnt
werden\ldots). Ob sich wenige Monolagen abgesetzt haben könnten, kann man durch diese
augenscheinliche Betrachtung nicht feststellen.\\
