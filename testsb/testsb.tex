Der Aufbau zur Spraydeposition wurde an einer kleineren HV-Apparatur getestet. \\(siehe
\ref{testapparatur}).
Der
Vorteil ist, dass diese Kammer wegen ihres kleineren Volumens schneller abgepumpt werden kann, was bei
auftretenden Problemen, bei denen man die Kammer öffnen muss, ein schnelleres Arbeiten ermöglicht.\\
Die sogenannte Testapparatur besteht aus einer Vorpumpe und  einer Hauptpumpe,
der Hauptkammer mit Massenspektrometer und Druckmessgerät sowie dem aufgesetzten
Sprühbeschichter. Am Würfel des Beschichters ist ein mobiler Pumpstand angeschlossen. Das Ventil
zum Einlassen des aufzutragenden Moleküls wird angesteuert über ein Netzteil von IotaOne. Das
Substrat, hier zum Test ein Glasträger, befindet sich unterhalb des Beschichters und kann durch einen
Fensterflansch beobachtet werden.\\
Die Ansteuerung des Ventils ermöglicht
prinzipiell zwei Modi: ein einmaliges Öffnen des
Ventils für einen einstellbaren Zeitraum (One
Shot Modus) und einen Zyklus von abwechselndem
Öffnen und Schließen des Ventils (Cycle Modus).
Da ein Einlassen der Lösung in das Vakuum
natürlich vorübergehend den Druck steigen lässt, wurde hier
zunächst der Cycle Modus getestet, da so
mutmaßlich der Druck durch das Schließen des
Ventils zwischendrin besser erhalten werden
kann. In diesem Modus müssen die Dauern für das
Öffnen (On-Time) und des Schließens (Off-Time)
gewählt werden, wobei die Off-Time
betriebsbedingt mindestens so lang wie die
On-Time sein muss.\\
Beim Vakuumieren der Apparatur wurde Ventil 1
 geschlossen und Beschichter sowie die 
 Hauptkammer getrennt gepumpt, bis beide
 Vakuumpumpen jeweils vollständig hochgefahren
 waren.
 Um ein Druckgefälle zwischen Würfel und unterem
 Teil des Aufbaus herzustellen, getrennt durch den
 Konus, wurde dann Ventil 2 geschlossen und
 Ventil 1 geöffnet.
 Somit sollte innerhalb des Konus und dem
 Verbindungsstück darunter annähernd der gleiche
 Druck herrschen wie im Rest der Haupkammer.
 Nun konnte das gelöste Molekül über eine Pipette
 in das Reservoir des elektrischen Ventils
 eingefüllt und die On- sowie Off-Time gewählt
 werden.\\
 
 
Bei dem zum Testen zunächst benutzten Molekül
handelte es sich um Kupfer-2-phthalocyanin (kurz
CuPc), ein blauer Feststoff, der als Pigment
beispielsweise in Kunststoffen, Lacken oder
Druckerfarben verwendet wird. Das CuPc wurde in
dem farblosen organischen Lösungsmittel
Dimethylsulfoxid (kurz DMSO) aufgelöst, woraus
sich eine grobe Suspension ergab.\\ 
Der Versuch, On- und Off-Zeiten für das Ventil im
Bereich von wenigen Mikrosekunden einzustellen,
scheiterte, vermutlich wegen der Trägheit der Mechanik des
Ventils. Nachfolgend wurden sukzessive
On-Zeiten von 20, 30, 40ms mit jeweils gleich
langen Off-Zeiten getestet. Die Zyklendauern
wurden von 5s langsam auf bis zu 90s erhöht.
Hier offenbarten sich direkt Probleme mit der
Suspension: obwohl sich zu Beginn immerhin die
Spitze des Konus blau färbte, wurde
zwischenzeitlich scheinbar durch gröbere Partikel
in der Lösung die Öffnung des Ventils verstopft,
sodass auch bei längerer Zyklusdauer der Pegel im
Ventil konstant blieb. Um Ablagerungen auf dem
Ventil zu entfernen, wurde es mit DMSO ausgespült
und danach für den nächsten Zyklus mit reinem
DMSO befüllt. Nach dem zweiten Durchgang öffnete
sich das Ventil wieder, der Pegel sank.
Vermutlich konnte sich die Öffnung des Ventils
durch die Vibrationen während des Betriebs
"`freischütteln"'. Bei den nachfolgenden
Messungen war nach einer Zyklusdauer von etwa 20s
das Ventil leer, sodass nur noch Luft in die
Kammer gesogen wurde.
Längere Zyklen konnten also nur durchgeführt
werden, indem manuell während der Messung das
Ventil nachgefüllt wurde.
\\
Der Druck in der Hauptkammer war zu Beginn jedes
Zyklus im Bereich von $10^{-7}$ bis $10^{-6}$.
War die Öffnung des Ventils frei und konnte die
Lösung in die Kammer eintreten, fiel der Druck in
der Hauptkammer in den Bereich $10^{-4}$. Eine
Druckmessung im Würfel war zunächst nicht möglich.\\
































