\section{Aufbau der Ultrahochvakuumsapparatur}\label{kapaufbau}

Die Spraydepositionsapparatur soll auf der bisher vorhandenen UHV-Apparatur (siehe Abb
\ref{uhvaufbau}) aufgebaut werden.~%, in der das Substrat präpariert werden kann.
 Diese besteht aus mehreren Kammern, die untereinander verbunden sind. Die gesamte Apparatur ist
auf vier Druckluftfüßen gelagert, um über den Boden übetragende störende Schwingungen zu minimieren. Um ein
Ultrahochvakuum im Bereich von \SI{e-10}{mbar} zu erzeugen, kann die Apparatur zunächst mit einem
mobilen Pumpstand %(hier verwendet: HiCube von Pfeiffer ) 
auf ein Hochvakuum (HV) im Bereich von
\SI{e-7}{mbar} gepumpt werden; ab einem Druck von \SI{e-6}{mbar} lässt sich eine fest installierte
Ionengetterpumpe hinzuschalten.

\begin{figure}[H]
\centering
\sffamily
\includesvg[svgpath=versuchsaufbau/]{uhv-apparatur}
\caption{\textit{Aufbau der UHV-Apparatur mit Molekül-, STM- und Hauptkammer. Unter der
Hauptkammer befindet sich die Ionengetterpumpe (hier nicht mit im Bild).}}
\label{uhvaufbau}
\end{figure}

Die Apparatur besteht im Wesentlichen aus drei wichtigen Kammern: 
der 
Molekülkammer, der STM-Kammer und der Hauptkammer. In der Molekülkammer kann die eingebrachte
Probe mit Hilfe des Molekülverdampfers (der in dieser Arbeit nicht verwendet wurde) bedampft werden.
Zukünftig soll auf diesen Teil der Kammer die im Rahmen dieser Bachelorarbeit gebaute
Spraydepositionsapparatur aufgesetzt werden. Dieser Teil der Kammer kann über zwei Ventile von den
anderen Kammern abgetrennt werden.
\\
In der STM-Kammer kann die Probe
mittels Rastertunnelmikroskopie und -spektroskopie untersucht werden. In der Haupkammer kann die
Probe durch einen unten in der Kammer eingebauten Metallverdampfer mit
verschiedenen Metallen (zur Zeit Gold, Eisen, Kobalt) bedampft, durch Erhitzen von Ablagerungen auf
der Oberfläche gereinigt ("`geflasht"') sowie die Probenoberfläche mit LEED untersucht werden. 
\\
Das verwendete Substrat kann im Probenhalter mittels eines
Transferstabs durch die Kammern transportiert werden. In die Molekülkammer kann es mit einem
weiteren, kleineren Transferstab gesetzt werden. Ein Transfer zum Probentisch des STMs in der
STM-Kammer ist mit Hilfe des sogenannten Wobblesticks möglich, mit dessen Greifzange die Probe an der Probennase gehalten und in
den STM-Tisch versetzt werden kann. In der Hauptkammer schlussendlich wird der Probenhalter in den
sogenannten Manipulator gesetzt, mit dem die Probe in alle Raumrichtungen verschoben sowie gedreht
werden kann. 
\\
Am Manipulator ist ein Heizfilament angebracht, über den der Kristall geflasht wird.
Dabei wird das Filament zum Glühen gebracht, die durch Glühemission emittierten Elektronen werden
durch eine an die Probe angelegte Hochspannung von \SI{700}{V} zum Kristall hin beschleunigt und
heizen diesen dann durch Stöße auf eine Temperatur von etwa \SI{2000}{K} auf. In diesem Prozess
werden aufgedampfte Schichten, aber auch Ablagerungen entfernt. Selbst bei einem Druck von \SI{e-10}{mbar} lagern sich
Restgasmoleküle auf Oberflächen ab; nach einem Tag entspricht dies in etwa einer Monolage.
\\
Weiterhin kann das Filament dazu benutzt werden, ohne angelegte Hochspannung die Probe einfach nur
zu erwärmen ("`tempern"'). Dies kann die Oberflächenstrukturen nach Aufdampfen eines Metalls
deutlich ändern. Die erreichte Temperatur von bis zu \SI{1000}{K} lässt sich aus dem Filamentstrom
bestimmen.
\\
 Über dem Metallverdampfer in der Hauptkammer befindet sich eine
Blende, sodass die darüber befindliche Probe durch Auf- oder Zuklappen je nach Bedarf bei
laufendem Verdampfer bedampft werden kann. Die Menge des verdampften Metalls wird über einen
Schwingquarz gemessen, der sich über dem Verdampfer befindet und somit mitbedampft wird. Anhand der
Änderung der Schwingfrequenz können Rückschlüsse auf die abgelagerte Masse auf dem Quarz gezogen
werden, worüber dann die Anzahl der aufgedampften Lagen auf der Probe bestimmt werden kann
\cite{Sau}.
Die Zunahme der Periodendauer (gemessen in "`Skalenteilen"', SKT) wird durch eine Bestimmung der
deponierten Menge auf dem Substrat mittels STM-Aufnahmen kalibriert. Nach einer solchen Kalibration
entsprechen 250 SKT etwa einer Monolage Gold auf dem Substrat.
\\
Der Druck in der Hauptkammer wird über eine Bayard-Alpart-Röhre gemessen, die sich unten in der
Kammer befindet. Zudem befindet sich ein Kaltkathodenvakuummeter an der Molekülkammer.
Zur Restgasanalyse und Lecksuche dient ein Quadrupolmassenspektrometer.


\section{Das Rastertunnelmikroskop}

Das hier verwendete Rastertunnelmikroskop ist ein Micro-STM der Firma Omicron. Es besteht
prinzipiell aus einem Röhrenscanner mit einer STM-Spitze aus Platin-Iridium-Draht und einem
Probentisch, über den der Probenhalter bewegt werden kann (siehe Abb. \ref{stmaufbau}). Sowohl der
Röhren\-scanner als auch der Probentisch werden über Piezoelemente bewegt, mit denen Scanner
wie auch Tisch in x-y-Richtung bewegt werden können, der Scanner zum Rastern der Oberfläche und der
Tisch zum Ausrichten der zu messenden Stelle der Probe; weiterhin kann der Scanner in z-Richtung an
die Probe angenähert werden.
\\
Die Positionierung des Probentisches sowie die  Grobannäherung der Spitze erfolgen über eine
Fernbedienung. Die Spitze des STMs wird dabei so nah wie möglich an die Probe herangefahren, was
über die Spiegelung der Spitze in der Probe, gefilmt über eine schräg unterhalb des Probentisches
angebrachte USB-Kamera, abgeschätzt werden kann. Die Feinanpassung wird über die zugehörige
Software erledigt.
Dabei wird die Spitze in kleinen Schritten soweit angenähert, bis ein Tunnelstrom gemessen werden
kann. Anschließen wird die Probe wiederum manuell noch zwei bis drei Schritte weiter an die Probe
herangefahren, um den Tunnelstrom zu stabilisieren.
\\

\begin{figure}[H]
\centering
\sffamily
\includesvg[svgpath=versuchsaufbau/]{stm-tisch}
\caption{\textit{a) Der Röhrenscanner mit Spitze, welche an die im STM-Tisch
liegende Probe angenähert ist (Bild von schräg unten). b) Der STM-Tisch mit der
Probe im Probenhalter von oben.}}
\label{stmaufbau}
\end{figure}

Grundsätzlich gibt es zwei Messmethoden. Im Konstanthöhenmodus wird der Abstand zwischen Spitze und
Probenoberfläche konstant gehalten. Die Messgröße ist der Tunnelstrom, der im resultierenden
Scanbild als Helligkeitskontrast dargestellt wird. Diese Einstellung ermöglicht eine relative hohe
Scangschwindigkeit, da die Höhe der Spitze nicht nachreguliert werden muss, eignet sich
allerdings auch nur für sehr ebenmäßige Oberflächen, da die Gefahr einer Kollision bei Erhöhungen
der Oberfläche oder die eines Verlusts des Tunnelkontakts bei Mulden besteht.
\\
Beim Konstantstrommodus wird der Tunnelstrom durch ständiges Nachregulieren der Höhe konstant
gehalten. So erhält man ein Höhenprofil für die abgerasterte x-y-Ebene, welches ebenfalls durch
Helligkeitskontraste dargestellt wird. Hierbei können auch unregelmäßige Oberflächen gescannt
werden. Zu bemerken ist, dass sich das dargestellte Höhenprofil zusammensetzt aus Topographie der
Oberfläche sowie konstanter elektronischer Zustandsdichte der Oberfläche.

\FloatBarrier

\section{Aufbau der Spraydepositionsapparatur}


 Die Spraydeposition funktioniert im Prinzip über ein Druckgefälle innerhalb ihres
Aufbaus, durch das sich die in einer Lösung befindlichen Moleküle durch Expansion durch die
Apparatur bewegen und auf dem Substrat im besten Falle haften bleiben. 
\\
\begin{figure}
\centering
\sffamily
\includesvg[svgpath=versuchsaufbau/]{wuerfel}
\caption{\textit{Der Aufbau der Spraydepositionsapparatur.
Der Würfel wird durch einen angeschlossenen Pumpstand auf etwa \SI{e-5}{mbar} vakuumiert, der untere
Teil des Aufbaus liegt mit Hilfe der Ionengetterpumpe im Bereich von \SI{e-8}{mbar}. Durch das
elektrisch gesteuerte Ventil ganz oben im Aufbau gelangt die Lösung in den Würfel; durch den
Druckunterschied wird sie weiter durch die Öffnung des Konus in den unteren Teil der Kammer gesogen
und bleibt auf der Probe haften. Ventil 1 dient zum Abtrennen des Würfels von der Molekülkammer, mit
Ventil 2 kann auch bei geschlossenem Ventil 1 das erste Verbindungsstück mitgepumpt werden.
}}
\label{sbaufbau}
\end{figure}
Ein Schema des Aufbaus ist in Abb. \ref{sbaufbau} zu sehen. Ganz oben befindet sich ein elektrisch
gesteuertes Nadelventil, in dessen Reservoir sich die Lösung befindet. Dieses Ventil wird über eine
Steuerungseinheit mit verschiedenen einstellbaren Öffnungsmodi bedient (siehe Kapitel
\ref{kaptest}).
Im Ventil befindet sich eine \SI{0,3}{mm} große, runde Öffnung, welches die Verbindung zwischen
Lösung und Vakuum darstellt. Durch ein System von Federn wird eine Nadel in dieses Loch gedrückt, um
es geschlossen zu halten.
Mit Hilfe eines Magnetfeldes kann die Nadel von der Öffnung gezogen und somit das Eindringen der
Lösung in den Würfel ermöglicht werden. 
Das Magnetfeld wird durch einen Elektromagneten erzeugt, welcher von der Steuerungseinheit des
Ventils gesteuert wird. Mit kurzen Stromimpulsen kann somit eine kurze Öffnungszeit realisiert
werden.
\\
Das Vakuum in diesem Würfel wird durch einen an einer Seite angeschlossenen Pumpstand hergestellt
und soll idealerweise im Bereich von \SI{e-5}{mbar} liegen. Im Würfel befindet sich ein metallischer
Konus, dessen Spitze eine Öffnung von wiederum \SI{0,1}{mm} aufweist. Er ist die Verbindung zwischen
Würfel und dem unteren Teil des Aufbaus, der auf der Molekülkammer aufgesetzt ist und einen Druck im Bereich von
\SI{e-8}{mbar} aufweisen sollte.  
Seine Aufgabe ist dafür zu sorgen, dass einerseits nur ein dünner Strahl der eingelassenen Lösung in
den unteren Teil des Aufbaus gelangt, damit sich die Moleküle gleichmäßig auf dem Substrat verteilen und
nicht in Tröpfchen. Gleichzeitig soll er durch seine kleine Öffnung einen schnellen Druckausgleich zwischen
dem unteren Teil und dem Würfel verhindern.
 \\
Überdies sind zwei weitere Ventile in dem Aufbau enthalten. Mit Ventil 1 kann der Würfel von der
Molekülkammer abgetrennt werden. Dies kann man nutzen, um den Würfel oder das elektrische Ventil zu
säubern, ohne dabei den kompletten vorderen Teil der Kammer belüften zu müssen. Wird der Würfel dann bei
geschlossenem Ventil 1 erneut über den Pumpstand vakuumiert, kann Ventil 2 geöffnet werden, damit auch das Verbindungsstück
unterhalb des Würfels über Schlauch und T-Stück mitgepumpt werden kann und nicht nur über die kleine Öffnung
des Konus.
\\
Das zweite Verbindungsstück ist eine Überbrückung zwischen Ventil 1 und Adapterflansch, mit dem
der \SI{40}{mm}-Anschluss von Ventil 1 auf den \SI{63}{mm}-Flansch der Molekülkammer gesetzt werden
kann. Eine direkt Verbindung zwischen diesem Ventil und dem Adapter ist nicht möglich, da beide ein
Sacklochgewinde besitzen.
\\
Unten in der Molekülkammer befindet sich das Substrat. Die durch den Konus in den unteren Teil des Aufbaus
gelangten Moleküle expandieren in der restlichen Kammer und sollten auf dem Substrat zum Liegen
kommen.

\FloatBarrier

